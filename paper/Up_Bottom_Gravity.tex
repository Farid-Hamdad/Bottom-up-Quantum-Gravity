\documentclass[11pt,a4paper]{article}

\usepackage[utf8]{inputenc}
\usepackage[T1]{fontenc}
\usepackage[french]{babel}
\usepackage{amsmath,amssymb,amsfonts}
\usepackage{physics}
\usepackage{hyperref}
\usepackage{geometry}
\usepackage{graphicx}
\usepackage{bm}
\usepackage{authblk}
\usepackage{setspace}
\usepackage{caption}
\usepackage{tcolorbox}
\usepackage{xcolor}
\usepackage{booktabs}
\usepackage{braket}

\geometry{margin=2.5cm}
\setstretch{1.15}

\definecolor{myblue}{RGB}{0,82,147}
\definecolor{mygreen}{RGB}{0,128,0}
\definecolor{myred}{RGB}{180,0,0}

% Figures live in repo_root/figures/, while this .tex is in repo_root/paper/
\graphicspath{{../figures/}}

\title{\textbf{Up\_Bottom Gravity:}\\
\vspace{0.3em}
\Large{Une Théorie de l'Émergence Gravitationnelle}\\
\vspace{0.3em}
\large{Espace, Temps et Gravité depuis l'Intrication Quantique}}

\author{Farid Hamdad}
\date{Février 2026}

\begin{document}

\maketitle

\vspace{1em}
\begin{center}
\itshape
L’intrication est une structure géométrique.\\
Ce que nous appelons espace est le motif collectif de ces corrélations.\\
Ce que nous appelons gravité est la thermodynamique de leur réorganisation.
\end{center}
\vspace{1.5em}

\begin{abstract}
\begin{tcolorbox}[colback=gray!10,colframe=myblue,title=Résumé Opérationnel]
Nous présentons le cadre \textbf{Up\_Bottom Gravity} dans lequel l'espace-temps et la gravité ne sont pas fondamentaux mais émergent de l'intrication quantique d'un état global $\ket{\Psi}$. Le temps est le flot modulaire des corrélations, l'espace est la géométrie informationnelle déduite de l'entropie d'intrication, et la gravité est la thermodynamique de leur réorganisation. Nous validons numériquement quatre prédictions: (1) une relation de type Jacobson $\delta S \simeq \delta E/T$, (2) une causalité effective compatible avec Lieb-Robinson, (3) l'émergence dimensionnelle $2\text{D} \to 3\text{D}$ via intrication non-locale, et (4) une signature opérationnelle de la correspondance ER=EPR (raccourcis géodésiques induits par l'intrication).
\end{tcolorbox}
\end{abstract}

%=============================================================================
\section{Ontologie Minimale: Le Postulat Fondamental}
%=============================================================================

\begin{tcolorbox}[colback=myblue!5,colframe=myblue,title=Postulat Unique]
Il existe un état quantique global pur $\ket{\Psi} \in \mathcal{H} = \bigotimes_{i \in V} \mathcal{H}_i$ où $V$ indexe un ensemble de degrés de liberté élémentaires (qubits).
\end{tcolorbox}

\textbf{Ce qui est absent:}
\begin{itemize}
    \item Aucun espace-temps préexistant
    \item Aucune métrique a priori
    \item Aucune structure causale externe
    \item Aucun champ classique
\end{itemize}

L'état $\ket{\Psi}$ est \textbf{atemporel} dans le sens où il n'évolue pas par rapport à un temps externe. C'est le ``bloc'' entier de l'existence quantique. Toute dynamique doit émerger de structures relationnelles \emph{internes} à $\ket{\Psi}$.

%=============================================================================
\section{Temps Émergent: Le Flot Modulaire}
%=============================================================================

\subsection{Partition et Horloge Interne}

On partitionne l'espace de Hilbert:
\begin{equation}
\mathcal{H} = \mathcal{H}_C \otimes \mathcal{H}_S
\end{equation}
où $C$ est une sous-système ``horloge'' et $S$ est le ``système'' (le reste).

\textbf{Contrainte de temps sans temps:}
\begin{equation}
(\hat{H}_C + \hat{H}_S)\ket{\Psi} = 0
\end{equation}
Cette contrainte exprime l'absence de temps externe : la dynamique est relationnelle.

\subsection{Flot Modulaire comme Temps Relationnel}

Pour toute région $A$:
\begin{equation}
\rho_A = \mathrm{Tr}_{\bar{A}} \ket{\Psi}\bra{\Psi}, \qquad K_A = -\log \rho_A
\end{equation}

Le \textbf{temps modulaire} $\tau$ est défini par:
\begin{equation}
\mathcal{O}(\tau) = e^{i K_A \tau} \mathcal{O} e^{-i K_A \tau}
\end{equation}

\textbf{Interprétation:} le temps n'est pas un paramètre externe mais un \textbf{paramètre interne} lié aux corrélations : une ``lecture'' relationnelle induite par le flot modulaire.

%=============================================================================
\section{Espace Émergent: Géométrie Informationnelle}
%=============================================================================

\subsection{Distance Informationnelle}

L'\textbf{information mutuelle} entre sites définit une proximité émergente:
\begin{equation}
I(i:j) = S(i) + S(j) - S(ij)
\end{equation}

On définit une distance informationnelle effective:
\begin{equation}
d_{ij} = -\log \frac{I(i:j)}{I_{\max}}
\end{equation}

Cette distance n'est pas nécessairement une métrique stricte (au sens de l'inégalité triangulaire), mais constitue une quantité opérationnelle : des degrés de liberté fortement corrélés sont géométriquement proches dans l'espace émergent reconstruit.

\subsection{Laplacien et Structure Géométrique}

Le graphe d'intrication a pour Laplacien:
\begin{equation}
L = D - W, \quad W_{ij} = I(i:j), \quad D_{ii} = \sum_j W_{ij}
\end{equation}

Dans la limite continue, ce Laplacien discret joue un rôle analogue à l'opérateur de Laplace-Beltrami.

\subsection{Loi d'Aire Holographique}

Pour des états fondamentaux gappés, l'entropie d'intrication admet typiquement une loi d'aire:
\begin{equation}
S(A) = \alpha |\partial A| + \gamma + \mathcal{O}(1/|\partial A|)
\end{equation}
signature d'un stockage de l'information sur la frontière, plutôt que dans le volume.

%=============================================================================
\section{Champs de Jauge Émergents: Structure Topologique}
%=============================================================================

\subsection{Code de Surface $Z_2$}

Sur un réseau discret, on peut implémenter un code topologique de type surface:
\begin{itemize}
    \item Opérateurs étoile: $A_v = \prod_{e \ni v} X_e$
    \item Opérateurs plaquette: $B_p = \prod_{e \in \partial p} Z_e$
\end{itemize}

Hamiltonien: $H_0 = -J_e \sum_v A_v - J_m \sum_p B_p$

L'état fondamental satisfait $\langle A_v \rangle = \langle B_p \rangle = 1$.

\subsection{Excitations et causalité effective}

Sous perturbations locales, les excitations peuvent se propager sous une borne de type Lieb-Robinson, suggérant une structure causale effective émergente.

%=============================================================================
\section{Gravité émergente : argument thermodynamique de Jacobson}
%=============================================================================

Dans le cadre \emph{Up\_Bottom Gravity}, la gravité n’est pas postulée comme une
interaction fondamentale, mais interprétée comme une manifestation
thermodynamique collective de la réorganisation de l’intrication quantique.
Cette idée s’inscrit dans la lignée de l’argument de Jacobson, selon lequel les
équations d’Einstein peuvent être comprises comme une équation d’état reliant
entropie, énergie et température.

\subsection{Principe thermodynamique}

L’argument de Jacobson repose sur l’hypothèse qu’une relation thermodynamique
locale de la forme
\begin{equation}
\delta S = \frac{\delta E}{T}
\end{equation}
est satisfaite pour toute région suffisamment petite, associée à un horizon causal local.
Sous cette hypothèse, les équations d’Einstein émergent comme condition de cohérence
thermodynamique.

Dans notre cadre, cette relation est reformulée de manière purement
informationnelle :
\begin{itemize}
    \item $S$ est l’entropie d’intrication associée à une sous-région $A$,
    \item $E$ est une énergie modulaire effective, définie par $E = \langle H \rangle$,
    \item $T$ est une température effective liée au flot modulaire, avec $T^{-1} = \beta = dS/dE$.
\end{itemize}
Aucune hypothèse d’espace-temps classique ni de champ gravitationnel n’est introduite à ce stade.

\subsection{Test numérique de la relation de Jacobson}

Nous testons numériquement la validité de la relation variationnelle
$\delta S \simeq \beta\,\delta E$ sur des états quantiques discrets de petite taille.
L’entropie d’intrication $S(A)$ est calculée à partir de la matrice de densité réduite,
tandis que l’énergie $E$ correspond à la valeur moyenne d’un Hamiltonien effectif
contrôlant les corrélations.

La Fig.~\ref{fig:jacobson} montre que l’entropie d’intrication varie de manière monotone
avec l’énergie effective.
De plus, les variations locales $\delta S$ et $\delta E$ présentent une corrélation
approximativement linéaire, indiquant que la relation thermodynamique de type Jacobson
est satisfaite à un bon niveau d’approximation.

Ces résultats constituent une validation numérique de principe de l’argument
thermodynamique, dans un cadre discret et sans géométrie préalable.

\subsection{Température modulaire et constante gravitationnelle effective}

La dérivée $\beta = dS/dE$ joue le rôle d’une température inverse effective,
associée au flot modulaire des corrélations.
Lorsque l’entropie satisfait une loi d’aire
\begin{equation}
S(A) = \alpha\,|\partial A| + \cdots ,
\end{equation}
la constante gravitationnelle effective est donnée par
\begin{equation}
G_{\mathrm{eff}} = \frac{1}{4\alpha}.
\end{equation}

Les estimations numériques de $G_{\mathrm{eff}}$ obtenues ici sont finies et cohérentes
sur une plage d’énergies, bien que dépendantes de la taille finie du système.
Elles doivent être interprétées comme des indicateurs qualitatifs de l’émergence
d’une dynamique gravitationnelle effective.

\subsection{Portée et limites}

Les résultats présentés ne constituent pas une dérivation complète des équations d’Einstein.
Ils montrent cependant que les conditions thermodynamiques nécessaires à l’argument de Jacobson
peuvent être satisfaites dans un cadre purement quantique, discret et informationnel.

\begin{figure}[h!]
\centering
\includegraphics[width=0.95\linewidth]{fig3_jacobson.png}
\caption{
\textbf{Relation thermodynamique de type Jacobson et gravité émergente.}
(\textit{Haut}) Entropie d’intrication $S(A)$ en fonction de l’énergie modulaire effective $E=\langle H\rangle$, montrant une relation monotone bien définie.
(\textit{Bas}) Test local de la relation variationnelle $\delta S \simeq \beta\,\delta E$, où $\beta = dS/dE$ joue le rôle d’une température inverse effective.
La corrélation observée indique que la dynamique gravitationnelle émergente peut être interprétée comme une thermodynamique de l’intrication, conformément à l’argument de Jacobson.
Les résultats présentés constituent une validation numérique de principe, limitée à des systèmes de petite taille.
}
\label{fig:jacobson}
\end{figure}

%=============================================================================
\section{Méthodes Numériques}
%=============================================================================

Cette section décrit explicitement les procédures numériques utilisées
pour reconstruire une géométrie émergente à partir de l’intrication quantique,
tester l’émergence dimensionnelle et analyser la correspondance ER=EPR.
L’objectif n’est pas l’optimisation algorithmique, mais la validation
opérationnelle du cadre conceptuel présenté.

\subsection{Système quantique considéré}

Nous considérons un système discret de $N$ qubits,
décrit par un état quantique global pur
\begin{equation}
\ket{\Psi} \in \mathcal{H} = \bigotimes_{i=1}^{N} \mathbb{C}^2 .
\end{equation}

Dans les simulations présentées ici, $N=9$ qubits disposés sur un réseau
$3 \times 3$ sont utilisés. Cette taille réduite permet un calcul exact
des matrices de densité réduites et des entropies de von Neumann,
tout en conservant une structure spatiale non triviale.
Les résultats doivent être compris comme des \emph{preuves de principe}.

\subsection{Calcul de l’information mutuelle}

Pour toute paire de qubits $(i,j)$, nous calculons l’information mutuelle
quantique définie par
\begin{equation}
I(i:j) = S(\rho_i) + S(\rho_j) - S(\rho_{ij}),
\end{equation}
où $S(\rho) = -\mathrm{Tr}(\rho \log \rho)$ est l’entropie de von Neumann,
$\rho_i = \mathrm{Tr}_{\bar{i}} \ket{\Psi}\bra{\Psi}$ et
$\rho_{ij} = \mathrm{Tr}_{\overline{ij}} \ket{\Psi}\bra{\Psi}$.

\subsection{Normalisation et seuil numérique}

L’information mutuelle est normalisée par sa valeur maximale :
\begin{equation}
\tilde{I}(i:j) = \frac{I(i:j)}{I_{\max}},
\qquad
I_{\max} = \max_{k,l} I(k:l).
\end{equation}

Pour éviter les divergences numériques liées au logarithme, un seuil $\epsilon$ est introduit :
\begin{equation}
\tilde{I}(i:j) \leftarrow \max(\tilde{I}(i:j), \epsilon),
\qquad \epsilon \sim 10^{-12}.
\end{equation}

\subsection{Distance informationnelle effective}

À partir de l’information mutuelle normalisée :
\begin{equation}
d_{ij} = -\log \tilde{I}(i:j).
\end{equation}
Cette distance capture opérationnellement l’idée centrale : des degrés de liberté
fortement intriqués sont géométriquement proches dans l’espace émergent reconstruit.

\subsection{Construction du graphe d’intrication}

Le système est représenté comme un graphe pondéré
$G = (V, E, W)$, où les sommets sont les qubits et les poids
$W_{ij} = I(i:j)$.
Le Laplacien est défini par
\begin{equation}
L = D - W,
\qquad
D_{ii} = \sum_j W_{ij}.
\end{equation}

\subsection{Embarquement géométrique par MDS}

À partir de la matrice des distances $D = (d_{ij})$, un algorithme de
Multi-Dimensional Scaling (MDS) reconstruit des points $\{x_i\}$ dans $\mathbb{R}^d$
minimisant l’erreur relative :
\begin{equation}
\epsilon_d =
\sqrt{
\frac{
\sum_{i<j} \left( \|x_i - x_j\| - d_{ij} \right)^2
}{
\sum_{i<j} d_{ij}^2
}
}.
\end{equation}

\subsection{Critère d’émergence dimensionnelle}

Une dimension $d$ est dite émergente si $\epsilon_d$ chute fortement par rapport à $\epsilon_{d-1}$
et se stabilise pour $d+1$, indiquant que l’augmentation de dimension n’apporte plus de gain
significatif.

\subsection{Test ER=EPR et géodésiques émergentes}

Pour une paire de sites $(i,j)$, nous comparons la séparation topologique sur le réseau initial
à la distance géodésique émergente $\|x_i-x_j\|$ dans l’espace reconstruit.
Une réduction drastique de la distance émergente pour une paire fortement intriquée est interprétée
comme une signature de canal informationnel de type \emph{wormhole-like}.

%=============================================================================
\section{Résultat central : émergence dimensionnelle}
%=============================================================================

L’un des résultats centraux de ce travail est la mise en évidence explicite du
caractère \emph{émergent} de la dimension spatiale.
Dans le cadre proposé, la dimension n’est ni postulée a priori, ni fixée par la
topologie du réseau, mais résulte collectivement de la structure de l’intrication quantique.

\subsection{Hypothèse physique}

\begin{quote}
\emph{Si des degrés de liberté quantiques topologiquement distants sont fortement
intriqués, alors une géométrie émergente de faible dimension ne peut plus
représenter fidèlement les corrélations, et une dimension spatiale effective
supplémentaire devient nécessaire.}
\end{quote}

Intuitivement, l’intrication non-locale agit comme une contrainte géométrique :
pour préserver la proximité informationnelle entre sites fortement corrélés,
l’espace émergent doit se courber ou se déplier dans une dimension plus élevée.

\subsection{Méthode de reconstruction}

À partir de $I(i:j)$, une distance $d_{ij}$ est construite puis un MDS reconstruit
une géométrie effective dans $\mathbb{R}^d$.
La dimension effective est identifiée comme la plus petite dimension $d$ pour
laquelle l’erreur $\epsilon_d$ devient faible et se stabilise.

\subsection{Résultats numériques}

Les résultats sont présentés sur la Fig.~\ref{fig:dimension}.
Pour un état à intrication locale, l’espace émergent est fidèlement représenté en 2D
et l’erreur ne diminue pas significativement au-delà de $d=2$.
Pour un état à intrication non-locale, une reconstruction 2D induit une distorsion importante,
et l’erreur chute lors du passage $d=2 \rightarrow 3$, indiquant une dimension effective supérieure.

\subsection{Interprétation physique}

La dimension spatiale apparaît comme une variable collective déterminée par la structure
des corrélations quantiques.
L’intrication non-locale agit comme une source de courbure informationnelle contraignant
l’espace émergent à s’étendre dans une dimension supplémentaire afin de préserver la proximité
entre degrés de liberté fortement corrélés.

\begin{figure}[h!]
\centering
\includegraphics[width=0.95\linewidth]{fig1_dimension_emergente.png}
\caption{
\textbf{Émergence dimensionnelle induite par l’intrication quantique.}
(\textit{Haut}) Reconstruction géométrique par Multi-Dimensional Scaling (MDS) d’un réseau de qubits $3\times3$.
À gauche, un état à intrication locale est fidèlement embarqué dans un espace bidimensionnel quasi plan.
À droite, un état à intrication non-locale nécessite une dimension spatiale supplémentaire pour être représenté sans distorsion.
(\textit{Bas}) Erreur relative de reconstruction $\epsilon_d$ en fonction de la dimension d’embarquement.
L’intrication non-locale induit une chute de l’erreur lors du passage de $d=2$ à $d=3$, indiquant que la dimension effective de l’espace émergent est imposée par la structure des corrélations quantiques.
}
\label{fig:dimension}
\end{figure}

%=============================================================================
\section{Correspondance ER = EPR : géodésiques d’intrication}
%=============================================================================

La correspondance ER = EPR suggère qu’une intrication quantique forte (EPR) entre deux sous-systèmes
se reflète, dans une description géométrique effective, par un pont de type Einstein–Rosen (ER).
Ici, nous testons cette idée de manière \emph{opérationnelle}, sans supposer de géométrie AdS/CFT
ni de dynamique gravitationnelle préalable.

\subsection{Principe opérationnel}

\begin{quote}
\emph{Si deux degrés de liberté quantiques sont fortement intriqués, alors la
distance géodésique entre eux dans l’espace émergent reconstruit à partir des
corrélations doit être fortement réduite, indépendamment de leur séparation topologique initiale.}
\end{quote}

Cette formulation ne postule pas un trou de ver classique, mais identifie une signature géométrique :
l’apparition d’un raccourci géodésique induit par l’intrication.

\subsection{Méthode de test}

Sur un réseau $3\times3$, nous comparons deux cas pour une paire topologiquement distante $(0,8)$ :
\begin{itemize}
    \item un état à intrication locale, $I(0,8)\ll I_{\max}$,
    \item un état à intrication non-locale, $I(0,8)\simeq I_{\max}$.
\end{itemize}
Nous comparons la distance topologique sur le réseau initial à la distance géodésique émergente
$\|x_0-x_8\|$ dans l’espace reconstruit.

\subsection{Résultats numériques}

Les résultats (Fig.~\ref{fig:erepr}) montrent que, dans l’état local, la distance émergente reste
comparable à la distance du réseau.
Dans l’état non-local, une intrication forte induit une réduction drastique de la distance géodésique :
malgré une séparation topologique importante, les deux sites deviennent géométriquement proches dans
l’espace reconstruit, signature \emph{wormhole-like} compatible avec ER = EPR.

\subsection{Interprétation et portée}

Ce résultat ne constitue pas une construction complète de trous de ver dynamiques au sens de la relativité générale.
Il met en évidence une \emph{signature géométrique opérationnelle} : correspondance entre intrication maximale
et raccourci géodésique dans une géométrie émergente reconstruite à partir des corrélations.

\begin{figure}[h!]
\centering
\includegraphics[width=0.95\linewidth]{fig2_er_epr.png}
\caption{
\textbf{Signature géométrique opérationnelle de la correspondance ER = EPR.}
Comparaison entre un état local (\textit{gauche}) et un état non-local (\textit{droite}) pour une paire de qubits topologiquement distants $(0,8)$.
Dans l’état local, l’information mutuelle est faible ($I(0,8)\ll I_{\max}$) et la distance géodésique dans l’espace émergent reconstruit reste comparable à la distance sur le réseau initial.
Dans l’état non-local, une intrication forte ($I(0,8)\simeq I_{\max}$) induit un raccourci géométrique drastique : la distance géodésique émergente devient quasi nulle malgré une séparation topologique importante.
Ce comportement constitue une signature géométrique mesurable d’un canal informationnel de type \emph{wormhole-like}, en accord avec l’idée ER = EPR.
}
\label{fig:erepr}
\end{figure}

%=============================================================================
\section{Synthèse: Chaîne d'Émergence Complète}
%=============================================================================

\begin{tcolorbox}[colback=mygreen!5,colframe=mygreen,title=Schéma Global]
\[
\ket{\Psi} \xrightarrow{\text{Réduction}} \rho_A \xrightarrow{-\log} K_A \xrightarrow{e^{iK_A\tau}} \textbf{Temps}
\]
\[
\ket{\Psi} \xrightarrow{\text{Corrélations}} I(i:j) \xrightarrow{-\log} d_{ij} \xrightarrow{\text{MDS}} g_{\mu\nu} \xrightarrow{\text{Jacobson}} \textbf{Gravité}
\]
\[
\ket{\Psi} \xrightarrow{\text{Non-local}} \text{Dimension émergente} \xrightarrow{\text{Raccourci géodésique}} \textbf{ER=EPR}
\]
\end{tcolorbox}

%=============================================================================
\section{Validations Numériques Réalisées}
%=============================================================================

\begin{enumerate}
    \item \textbf{Relation de type Jacobson:} test variationnel $\delta S \simeq \beta\,\delta E$ (preuve de principe) et émergence d'une échelle effective.
    \item \textbf{Causalité effective:} compatibilité qualitative avec une borne de type Lieb-Robinson dans des modèles topologiques discrets.
    \item \textbf{Émergence dimensionnelle:} transition $2\text{D} \rightarrow 3\text{D}$ sous intrication non-locale (Fig.~\ref{fig:dimension}).
    \item \textbf{ER=EPR:} intrication forte $\Rightarrow$ raccourci géodésique \emph{wormhole-like} (Fig.~\ref{fig:erepr}).
\end{enumerate}

%=============================================================================
\section{Limites et Perspectives}
%=============================================================================

\subsection{Limites Actuelles}
\begin{itemize}
    \item Taille système: preuve de principe sur petites tailles (effets de bord possibles)
    \item Dimension: reconstruction euclidienne effective (pas une mesure complète de dimension fractale)
    \item Dynamique: analyse principalement statique (états)
\end{itemize}

\subsection{Extensions Proposées}
\begin{itemize}
    \item \textbf{Tensor networks / DMRG:} extension à $N \gtrsim 25$ pour tester le scaling et stabiliser $G_{\mathrm{eff}}$
    \item \textbf{Courbure discrète (Ricci):} estimation contrôlée sur la géométrie reconstruite
    \item \textbf{Expérimental:} implémentations sur plateformes qubits (supraconducteurs, ions piégés)
    \item \textbf{ER=EPR dynamique:} étude de scénarios modulaires et évolution des raccourcis géodésiques
\end{itemize}

%=============================================================================
\section{Conclusion}
%=============================================================================

\begin{tcolorbox}[colback=myred!5,colframe=myred,title=Maxime Fondamentale]
\begin{center}
\Large
L'univers fondamental est un état quantique atemporel.\\
L'espace est la carte de ses corrélations.\\
Le temps est la lecture interne de ces corrélations.\\
La gravité est la thermodynamique de leur réorganisation.
\end{center}
\end{tcolorbox}

Dans ce cadre, la physique peut être vue comme l'étude de la manière dont l'espace-temps et les lois effectives émergent
d'une structure relationnelle quantique fondamentale.
Les validations numériques présentées constituent des preuves de principe : elles montrent que des signatures géométriques
et thermodynamiques attendues (dimension effective, raccourcis géodésiques, relation de type Jacobson) peuvent être obtenues
à partir de corrélations d'intrication dans des systèmes discrets.

%=============================================================================
\begin{thebibliography}{9}
\bibitem{jacobson} T. Jacobson, ``Thermodynamics of Spacetime: The Einstein Equation of State,'' \textit{Phys. Rev. Lett.} 75, 1260 (1995).
\bibitem{er=epr} J. Maldacena and L. Susskind, ``Cool Horizons for Entangled Black Holes,'' \textit{Fortschritte der Physik} 61, 781 (2013).
\bibitem{vanraamsdonk} M. Van Raamsdonk, ``Building up Spacetime with Quantum Entanglement,'' \textit{Int. J. Mod. Phys. D} 19, 2429 (2010).
\bibitem{ryu-takayanagi} S. Ryu and T. Takayanagi, ``Holographic Derivation of Entanglement Entropy from AdS/CFT,'' \textit{Phys. Rev. Lett.} 96, 181602 (2006).
\end{thebibliography}

\end{document}
