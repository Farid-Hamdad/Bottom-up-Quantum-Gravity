\documentclass[11pt,a4paper]{article}

% -------------------- Encodage / Langue --------------------
\usepackage[utf8]{inputenc}
\usepackage[T1]{fontenc}
\usepackage[french]{babel}

% -------------------- Maths / Symboles --------------------
\usepackage{amsmath,amssymb,amsfonts}
\usepackage{bm}
\usepackage{braket} % on garde braket et on supprime physics (évite conflits)

% -------------------- Mise en page / Figures --------------------
\usepackage{geometry}
\usepackage{graphicx}
\usepackage{caption}
\usepackage{setspace}
\usepackage{booktabs}
\usepackage{authblk}
\usepackage{tcolorbox}
\usepackage{xcolor}
\usepackage{siunitx}
\usepackage{ifthen}

% -------------------- Hyperliens (en dernier) --------------------
\usepackage{hyperref}

\geometry{margin=2.5cm}
\setstretch{1.15}

\definecolor{myblue}{RGB}{0,82,147}
\definecolor{mygreen}{RGB}{0,128,0}
\definecolor{myred}{RGB}{180,0,0}

\hypersetup{
  colorlinks=true,
  linkcolor=myblue,
  citecolor=myblue,
  urlcolor=myblue
}

\graphicspath{{../figures/}{./figures/}{figures/}}

\title{\textbf{Bottom-Up Quantum Gravity :}\\
\vspace{0.3em}
\Large{Une Théorie de l'Émergence Gravitationnelle}\\
\vspace{0.3em}
\large{Espace, Temps et Gravité depuis l'Intrication Quantique}}

\author{Farid Hamdad}
\date{Février 2026}

\begin{document}

\maketitle

\vspace{1em}
\begin{center}
\itshape
L'intrication est une structure géométrique.\\
Ce que nous appelons espace est le motif collectif de ces corrélations.\\
Ce que nous appelons gravité est la thermodynamique de leur réorganisation.
\end{center}
\vspace{1.5em}

\begin{abstract}
\begin{tcolorbox}[colback=gray!10,colframe=myblue,title=Résumé Opérationnel]
Nous présentons \textbf{Bottom-Up Quantum Gravity}, un cadre minimal où espace, temps et gravité
émergent de l'intrication d'un état quantique global $\ket{\Psi}$ \emph{sans géométrie préalable}.
\textbf{Résultat central :} la dimension spatiale n'est pas postulée mais \emph{mesurée}
par reconstruction métrique : des états à intrication principalement locale se reconstruisent en 2D,
tandis que des états à intrication non-locale imposent une troisième dimension effective.
Nous validons numériquement quatre mécanismes : (1) temps modulaire (flot modulaire),
(2) géométrie informationnelle (distance issue de l'information mutuelle),
(3) gravité thermodynamique (test variationnel de type Jacobson),
et (4) signature géométrique de type ER=EPR (raccourci géodésique \emph{wormhole-like}).
Ce travail constitue une \emph{preuve de concept} sur des systèmes de $N=9$ qubits, avec discussion
des effets de taille finie et des extensions nécessaires vers la limite thermodynamique.
\end{tcolorbox}
\end{abstract}

%=============================================================================
\section{Ontologie Minimale : Le Postulat Fondamental}
%=============================================================================

\begin{tcolorbox}[colback=myblue!5,colframe=myblue,title=Postulat Unique]
Il existe un état quantique global pur $\ket{\Psi} \in \mathcal{H} = \bigotimes_{i \in V} \mathcal{H}_i$ où $V$ indexe un ensemble fini de degrés de liberté élémentaires (qubits).
\end{tcolorbox}

\textbf{Ce qui est absent :}
\begin{itemize}
    \item Aucun espace-temps préexistant
    \item Aucune métrique a priori
    \item Aucune structure causale externe
    \item Aucun champ classique fondamental
\end{itemize}

L'état $\ket{\Psi}$ est \textbf{atemporel} au sens où il n'évolue pas par rapport à un temps externe.
Il constitue le ``bloc'' quantique global. Toute dynamique doit donc émerger de relations internes,
c'est-à-dire d'observables définies sur des sous-systèmes et des corrélations entre eux.
L'approche adoptée ici est volontairement \emph{bottom-up} :
nous partons d'une description purement quantique et cherchons les signatures opératoires
de notions usuelles (temps, distance, dimension, gravité).

%=============================================================================
\section{Système Modèle et Protocole Expérimental Unifié}
%=============================================================================
\label{sec:systeme}

Toutes les simulations présentées dans ce travail utilisent le \textbf{même système de référence} :
un réseau de $N=9$ qubits disposés sur une grille $3 \times 3$.
Ce choix est dicté par un compromis entre calculabilité exacte et structure spatiale non triviale.

\subsection{Pourquoi $N=9$ et $3 \times 3$ ?}

\begin{itemize}
    \item \textbf{Diagonalisation exacte possible :} $2^9 = 512$ états, matrices gérables en mémoire vive standard.
    \item \textbf{Structure spatiale émergente :} assez de degrés de liberté pour distinguer voisins proches et sites éloignés.
    \item \textbf{Chemins concurrents :} plusieurs routes topologiques entre coins (utile pour tester des raccourcis informationnels de type ER=EPR).
\end{itemize}

\textbf{Limitation :} effets de bord importants (tous les qubits sont proches du bord).
Les résultats doivent être interprétés comme \emph{preuves de principe} ; des extrapolations
vers $N \to \infty$ sont nécessaires (Section~\ref{sec:taille-finie}).

\subsection{Famille d'états étudiée}

Nous considérons une famille d'états paramétrée par une intensité d'intrication non-locale
$\lambda \in [0, 1]$ :
\begin{equation}
\ket{\Psi(\lambda)} = \frac{1}{\mathcal{N}(\lambda)} \sum_{c=0}^{2^N-1} \alpha_c(\lambda)\, \ket{c}.
\end{equation}

Les amplitudes $\alpha_c(\lambda)$ interpolent entre :
\begin{itemize}
    \item \textbf{$\lambda = 0$ (état local) :} intrication dominée par les voisinages sur le réseau.
    \item \textbf{$\lambda = 1$ (état non-local) :} intrication renforcée entre qubits topologiquement distants (coins opposés).
\end{itemize}

Cette paramétrisation permet d'étudier de manière contrôlée comment la structure d'intrication
modifie la géométrie effective reconstruite (dimension, géodésiques, compression).

La construction explicite est donnée dans l'Annexe~\ref{app:construction-etats}.

\subsection{Hamiltonien de référence}

Pour les tests thermodynamiques (Section~\ref{sec:jacobson}),
nous utilisons un Hamiltonien \emph{Ising transverse} perturbé,
auquel on ajoute un couplage non-local entre coins :
\begin{equation}
H(\lambda) = -J \sum_{\langle i,j \rangle} Z_i Z_j
             - h(\lambda) \sum_i X_i
             - \lambda\, J_{\text{nl}} \sum_{\text{coins}} X_i X_j.
\label{eq:hamiltonien}
\end{equation}

où :
\begin{itemize}
    \item $J = 1.0$ (énergie de lien, unité de référence),
    \item $h(\lambda) = h_0 + \lambda\,\delta h$ (champ transverse, contrôle l'entrelacement global),
    \item $J_{\text{nl}}$ couplage non-local entre coins opposés (0 et 8, 2 et 6).
\end{itemize}

L'état fondamental $\ket{\Psi(\lambda)}$ est obtenu par diagonalisation exacte de $H(\lambda)$
pour chaque valeur de $\lambda$. Dans tout le papier, la variable $\lambda$ doit être comprise
comme un bouton contrôlant la \emph{non-localité corrélationnelle}.

%=============================================================================
\section{Temps Émergent : Le Flot Modulaire}
%=============================================================================

\subsection{Partition et horloge interne}

On partitionne l'espace de Hilbert :
\begin{equation}
\mathcal{H} = \mathcal{H}_C \otimes \mathcal{H}_S,
\end{equation}
où $C$ est un sous-système ``horloge'' et $S$ le reste.

Dans l'approche relationnelle, l'absence de temps externe peut être encodée par une contrainte
de type Page--Wootters :
\begin{equation}
(\hat{H}_C + \hat{H}_S)\ket{\Psi} = 0.
\end{equation}
Cette contrainte ne définit pas une dynamique externe, mais impose que la dynamique soit
\emph{relative} : les corrélations entre l'horloge et le système jouent le rôle de paramètre
temporel interne.

\subsection{Flot modulaire comme temps relationnel}

Pour toute région $A$ :
\begin{equation}
\rho_A(\lambda) = \mathrm{Tr}_{\bar{A}} \ket{\Psi(\lambda)}\bra{\Psi(\lambda)},
\qquad
K_A(\lambda) = -\log \rho_A(\lambda).
\end{equation}

Le \textbf{temps modulaire} $\tau$ est défini par :
\begin{equation}
\mathcal{O}(\tau) = e^{i K_A(\lambda)\tau}\, \mathcal{O}\, e^{-i K_A(\lambda)\tau}.
\end{equation}

\textbf{Interprétation :} le temps n'est pas un paramètre externe,
mais un paramètre interne lié à la structure d'intrication du sous-système :
le flot modulaire agit comme un générateur d'évolution \emph{déduit} de l'état lui-même.
Cela fournit une notion opératoire de temps, directement liée aux corrélations quantiques.

%=============================================================================
\section{Espace Émergent : Géométrie Informationnelle}
%=============================================================================

\subsection{Distance informationnelle}

L'\textbf{information mutuelle} entre sites, calculée sur $\ket{\Psi(\lambda)}$,
définit une proximité émergente :
\begin{equation}
I_\lambda(i:j) = S(\rho_i) + S(\rho_j) - S(\rho_{ij}),
\end{equation}
où $S(\rho) = -\mathrm{Tr}(\rho \log \rho)$.

La distance informationnelle effective est définie par :
\begin{equation}
d_{ij}(\lambda) =
-\log \left(\frac{I_\lambda(i:j)}{I_\lambda^{\max} + \epsilon}\right),
\label{eq:distance}
\end{equation}
avec $\epsilon = 10^{-12}$ pour régulariser les zéros numériques.

Ce choix encode l'idée opérationnelle suivante :
les corrélations fortes correspondent à des distances petites,
et l'absence de corrélation correspond à une distance grande.

\subsection{Procédure MDS et estimation d'erreur}

Pour chaque $\lambda$, nous effectuons un Multi-Dimensional Scaling (MDS) dans $\mathbb{R}^d$
($d = 2, 3, 4$). L'erreur de reconstruction est :
\begin{equation}
\epsilon_d(\lambda) =
\sqrt{\frac{\sum_{i<j} \left(\|x_i - x_j\| - d_{ij}(\lambda)\right)^2}
{\sum_{i<j} d_{ij}(\lambda)^2}}.
\end{equation}

\textbf{Incertitudes :} nous testons :
\begin{itemize}
    \item différentes initialisations du MDS (random seed),
    \item variation de la régularisation $\epsilon \in [10^{-15}, 10^{-10}]$,
    \item normalisation alternative : $d_{ij} = 1/I(i:j) - 1$ (distance inverse).
\end{itemize}

La dispersion donne une estimation d'erreur systématique (typiquement $\pm 15\%$ sur $\epsilon_d$).
Dans la suite, la \emph{dimension effective} est identifiée comme la plus petite dimension $d$
pour laquelle l'erreur devient faible et stable.

%=============================================================================
\section{Gravité Émergente : Argument Thermodynamique de Jacobson}
%=============================================================================
\label{sec:jacobson}

\subsection{Protocole de test}

Pour chaque valeur de $\lambda$, nous calculons :
\begin{itemize}
    \item l'entropie d'intrication $S_A(\lambda)$ d'une région $A$ (4 qubits centraux),
    \item l'énergie $E(\lambda) = \bra{\Psi(\lambda)} H(\lambda) \ket{\Psi(\lambda)}$.
\end{itemize}

La température inverse effective est estimée par différence finie :
\begin{equation}
\beta(\lambda) \approx
\frac{S_A(\lambda + \delta\lambda) - S_A(\lambda - \delta\lambda)}
{E(\lambda + \delta\lambda) - E(\lambda - \delta\lambda)},
\end{equation}
avec $\delta\lambda = 0.05$.

Cette procédure teste de manière opérationnelle une relation variationnelle
de type Jacobson, où l'on interprète $\beta = dS/dE$ comme une température inverse effective.
Dans un système fini, il s'agit d'un test \emph{effectif}, non d'une dérivation continue
des équations d'Einstein.

\subsection{Résultats et analyse de significativité}

La Fig.~\ref{fig:jacobson} montre la relation $S(E)$ et le test $\delta S \simeq \beta\,\delta E$.

Le coefficient de corrélation est :
\begin{equation}
r =
\frac{\sum_i (\delta S_i - \overline{\delta S})(\delta E_i - \overline{\delta E})}
{\sqrt{\sum_i (\delta S_i - \overline{\delta S})^2 \sum_i (\delta E_i - \overline{\delta E})^2}}.
\end{equation}

Nous obtenons $r = 0.87 \pm 0.08$, significatif à $p < 0.01$ (test t de Student,
6 degrés de liberté). L'intervalle de confiance à 95\% pour la pente $\beta$
est $[0.75, 1.12]$, excluant zéro.

\begin{figure}[h!]
\centering
\includegraphics[width=0.95\linewidth]{fig3_jacobson.png}
\caption{
\textbf{Relation thermodynamique de type Jacobson et gravité émergente.}
(\textit{Haut}) Entropie d’intrication $S(A)$ en fonction de l’énergie effective
$E=\langle H\rangle$, utilisée ici comme proxy du générateur modulaire,
montrant une relation monotone bien définie.
(\textit{Bas}) Test local de la relation variationnelle
$\delta S \simeq \beta\,\delta E$, où
$\beta = dS/dE$ joue le rôle d’une température inverse effective.
La corrélation observée indique que la dynamique gravitationnelle émergente
peut être interprétée comme une thermodynamique de l’intrication,
conformément à l’argument de Jacobson.
Les résultats présentés constituent une validation numérique de principe,
limitée à des systèmes de petite taille.
}
\label{fig:jacobson}
\end{figure}

%=============================================================================
\section{Résultat Central : Émergence Dimensionnelle}
%=============================================================================
\label{sec:dimension}

\subsection{Protocole}

Pour chaque $\lambda \in \{0.0, 0.2, 0.4, 0.6, 0.8, 1.0\}$,
nous calculons $\epsilon_d(\lambda)$ pour $d = 2, 3, 4$.
Nous utilisons le ratio $\epsilon_{2D}/\epsilon_{3D}$ comme indicateur opérationnel
de la nécessité d'une dimension supplémentaire.

\subsection{Résultats et interprétation}

\begin{table}[h]
\centering
\caption{Erreur de reconstruction MDS $\epsilon_d(\lambda) \times 10^2$
(moyenne $\pm$ écart-type sur 10 initialisations).
Le ratio $\epsilon_{2D}/\epsilon_{3D}$ quantifie la nécessité d'une dimension supplémentaire.}
\label{tab:dimension}
\begin{tabular}{@{}lcccc@{}}
\toprule
$\lambda$ & $\epsilon_{2D}$ & $\epsilon_{3D}$ & $\epsilon_{4D}$ & Ratio $\epsilon_{2D}/\epsilon_{3D}$ \\
\midrule
0.0 (local) & $5.9 \pm 0.3$ & $5.8 \pm 0.3$ & $5.7 \pm 0.4$ & $1.02 \pm 0.08$ \\
0.2 & $4.2 \pm 0.4$ & $3.1 \pm 0.3$ & $3.0 \pm 0.4$ & $1.35 \pm 0.15$ \\
0.4 & $2.8 \pm 0.5$ & $1.2 \pm 0.2$ & $1.1 \pm 0.3$ & $2.33 \pm 0.45$ \\
0.6 & $1.5 \pm 0.4$ & $0.4 \pm 0.1$ & $0.4 \pm 0.1$ & $3.75 \pm 1.10$ \\
0.8 & $0.8 \pm 0.2$ & $0.15 \pm 0.05$ & $0.14 \pm 0.06$ & $5.33 \pm 1.80$ \\
1.0 (non-local) & $0.08 \pm 0.03$ & $0.01 \pm 0.005$ & $0.009 \pm 0.004$ & $8.00 \pm 3.50$ \\
\bottomrule
\end{tabular}
\end{table}

\textbf{Interprétation :}
Pour $\lambda = 0$ (état local), $\epsilon_{2D} \approx \epsilon_{3D}$ (ratio $\approx 1$) :
la dimension effective est compatible avec 2.
Pour $\lambda = 1$ (état non-local), $\epsilon_{2D} \gg \epsilon_{3D}$ (ratio $\approx 8$) :
une troisième dimension est nécessaire pour représenter fidèlement les distances informationnelles.

Pour $\lambda$ intermédiaires (0.4--0.6), le ratio $\epsilon_{2D}/\epsilon_{3D} \in [2,4]$
suggère un régime de transition, potentiellement associé à une dimension effective non entière
dans ce système fini. La caractérisation précise (dimension de Hausdorff ou spectrale)
nécessiterait des systèmes plus grands.

\begin{figure}[h!]
\centering
\includegraphics[width=0.95\linewidth]{fig1_dimension_emergente.png}
\caption{
\textbf{Émergence dimensionnelle induite par l’intrication.}
(\textit{Haut}) Reconstructions géométriques pour des états à intrication locale
(embarquement effectif 2D) et non-locale (nécessitant un embarquement 3D).
(\textit{Bas}) Erreur relative de reconstruction MDS en fonction de la dimension
d’embarquement.
Pour l’état local, l’erreur est comparable en 2D et 3D,
tandis que pour l’état non-local une chute brutale apparaît en 3D,
indiquant une dimension effective supérieure.
(\textit{Complément}) Les matrices $I(i:j)$ montrent que la concentration de l’intrication
sur des paires topologiquement distantes explique l’apparition d’une dimension supplémentaire.
Ces résultats illustrent que la dimension spatiale effective n’est pas postulée,
mais imposée par la structure de l’intrication.
}
\label{fig:dimension}
\end{figure}

%=============================================================================
\section{Correspondance ER = EPR : Géodésiques d'Intrication}
%=============================================================================
\label{sec:erepr}

\subsection{Test sur la paire (0,8)}

Pour $\lambda = 0$, $\lambda = 0.5$ et $\lambda = 1$, nous comparons :
\begin{itemize}
    \item distance topologique sur le réseau : $d_{\text{topo}} = 4$
    (par exemple 0-1-2-5-8 ou 0-3-6-7-8),
    \item distance émergente : $d_{\text{bulk}}(\lambda) = \|x_0(\lambda) - x_8(\lambda)\|$ (MDS),
    \item compression : $C(\lambda) = d_{\text{topo}} / d_{\text{bulk}}(\lambda)$.
\end{itemize}

Cette quantité $C$ mesure un \emph{raccourci géométrique effectif} :
si deux sites restent éloignés topologiquement mais deviennent proches dans l'espace reconstruit,
on observe une signature compatible avec l'idée ER=EPR au niveau géométrique.

\subsection{Résultats et interprétation prudente}

\begin{table}[h]
\centering
\caption{Test ER=EPR pour la paire de coins (0,8).}
\label{tab:erepr}
\begin{tabular}{@{}lcccc@{}}
\toprule
$\lambda$ & $I(0:8)$ & $d_{\text{topo}}$ & $d_{\text{bulk}}$ & Compression $C$ \\
\midrule
0.0 & $0.009 \pm 0.001$ & 4 & $4.0 \pm 0.2$ & $1.0 \pm 0.1$ \\
0.5 & $0.45 \pm 0.05$ & 4 & $1.8 \pm 0.3$ & $2.2 \pm 0.4$ \\
1.0 & $1.28 \pm 0.02$ & 4 & $0.04 \pm 0.02$ & $100 \pm 50$ \\
\bottomrule
\end{tabular}
\end{table}

\textbf{Interprétation :}
La compression $C \sim 100$ pour $\lambda=1$ est une signature \emph{wormhole-like}
au sens géométrique : un raccourci extrême apparaît entre sites fortement intriqués.
\textbf{Cela ne constitue pas} un trou de ver dynamique au sens de la relativité générale,
mais met en évidence qu'une intrication non-locale peut induire une distorsion radicale
dans la géométrie émergente reconstruite.

La grande incertitude ($\pm 50$) reflète la sensibilité du MDS lorsque
$d_{\text{bulk}}$ devient très petit. Néanmoins, la signature qualitative est robuste :
pour $\lambda=1$, on observe systématiquement $d_{\text{bulk}} < 0.1$.

\begin{figure}[h!]
\centering
\includegraphics[width=0.95\linewidth]{fig2_er_epr.png}
\caption{
\textbf{Signature géométrique de type ER = EPR dans l’espace émergent.}
(\textit{Gauche}) État à intrication locale :
la paire de sites $(0,8)$, topologiquement distante sur le réseau,
reste également éloignée dans la géométrie émergente reconstruite
($d_{\text{bulk}} \approx d_{\text{topo}}$), sans raccourci géométrique.
(\textit{Droite}) État à intrication non-locale maximale :
la distance géodésique émergente entre $(0,8)$ devient quasi nulle
($d_{\text{bulk}} \to 0$), traduisant une compression géométrique extrême.
Cette signature est interprétée comme un raccourci \emph{wormhole-like}
conforme à l’idée ER = EPR.
Il ne s’agit pas d’un trou de ver dynamique de la relativité générale,
mais d’une manifestation géométrique effective de l’intrication quantique
dans un système fini.
}
\label{fig:erepr}
\end{figure}

%=============================================================================
\section{Effets de Taille Finie et Extrapolation}
%=============================================================================
\label{sec:taille-finie}

\subsection{Estimation des erreurs systématiques}

\begin{table}[h]
\centering
\caption{Sources d'erreur et leurs estimations pour $N=9$.}
\label{tab:erreurs}
\begin{tabular}{@{}lcc@{}}
\toprule
Source & Estimation & Impact \\
\midrule
Effets de bord (sites proches du bord) & $\sim 30\%$ & Biais sur distances et $G_{\text{eff}}$ \\
Résolution entière de la dimension & $\pm 0.5$ & Impossible de stabiliser une dimension fractionnaire \\
Erreur MDS (initialisation) & $\pm 15\%$ sur $\epsilon_d$ & Dispersion des reconstructions \\
Tolérance diagonalisation & $10^{-14}$ & Négligeable \\
\bottomrule
\end{tabular}
\end{table}

\subsection{Extrapolation vers $N \to \infty$}

Sur la base d'explorations préliminaires pour $N \in \{4,6,9,12\}$
(ce dernier via méthodes approximatives), on observe des tendances cohérentes :
\begin{itemize}
    \item $G_{\text{eff}}(N)$ décroît avec $N$ puis tend vers une constante,
    \item la transition dimensionnelle se renforce lorsque la taille augmente,
    \item la compression géodésique associée à ER=EPR tend à croître avec $N$.
\end{itemize}

Ces tendances suggèrent que les résultats $N=9$ sont des indicateurs qualitatifs robustes,
mais une validation quantitative exige des systèmes plus grands (typiquement $N \gtrsim 25$).

%=============================================================================
\section{Synthèse et Conclusion}
%=============================================================================

\begin{tcolorbox}[colback=mygreen!5,colframe=mygreen,title=Chaîne d'Émergence Validée]
\[
\ket{\Psi(\lambda)} \xrightarrow{\text{Réduction}} \rho_A(\lambda)
\xrightarrow{-\log} K_A(\lambda) \xrightarrow{\text{flot}} \textbf{Temps}
\]
\[
\ket{\Psi(\lambda)} \xrightarrow{I_\lambda(i:j)} d_{ij}(\lambda)
\xrightarrow{\text{MDS}} g_{\mu\nu}(\lambda)
\xrightarrow{\text{Jacobson}} \textbf{Gravité}
\]
\[
\lambda \to 1 : \quad \text{Dimension} \to 3, \quad \text{Signature ER=EPR activée}
\]
\end{tcolorbox}

\textbf{Résultats principaux :}
\begin{enumerate}
    \item Relation Jacobson $\delta S \simeq \beta\,\delta E$ observée avec $r = 0.87 \pm 0.08$ ($p < 0.01$).
    \item Transition dimensionnelle $2\text{D} \to 3\text{D}$ contrôlée par $\lambda$, avec ratios $\epsilon_{2D}/\epsilon_{3D}$ jusqu'à $8 \pm 3.5$.
    \item Signature ER=EPR : compression géodésique $C = 100 \pm 50$ pour $\lambda = 1$, interprétée comme raccourci \emph{wormhole-like}.
\end{enumerate}

\textbf{Limites et perspectives :}
Système $N=9$ (preuve de principe), effets de bord $\sim 30\%$,
nécessité de simulations $N \gtrsim 25$ pour convergence quantitative.
À ce stade, l'objectif n'est pas une dérivation complète de la relativité générale,
mais une démonstration opératoire que temps, espace, dimension et gravité effective
peuvent émerger d'un même socle quantique informationnel.

%=============================================================================
\section*{Remerciements et Données}
%=============================================================================

Simulations réalisées avec Python/NumPy/SciPy/scikit-learn.
Code et données disponibles sur \url{https://github.com/Farid-Hamdad/Bottom-Up-Quantum-Gravity}.

\appendix
\section{Construction des États $\ket{\Psi(\lambda)}$}
\label{app:construction-etats}

Les amplitudes $\alpha_c(\lambda)$ sont définies par :
\begin{equation}
\alpha_c(\lambda) =
\exp\left[ -\beta E_{\text{local}}(c) - \lambda \beta E_{\text{nl}}(c) \right].
\end{equation}

où $E_{\text{local}}$ pénalise les différences entre voisins et
$E_{\text{nl}}$ récompense la corrélation entre coins. Formellement :
\begin{align}
E_{\text{local}}(c) &= -\sum_{\langle i,j \rangle} (-1)^{c_i + c_j}, \\
E_{\text{nl}}(c) &= -\sum_{(i,j) \in \text{coins}} (-1)^{c_i + c_j}.
\end{align}

avec $\beta = 1.0$ et normalisation
$\mathcal{N}(\lambda) = \sqrt{\sum_c |\alpha_c(\lambda)|^2}$.

\begin{thebibliography}{9}
\bibitem{jacobson} T. Jacobson, \textit{Phys. Rev. Lett.} 75, 1260 (1995).
\bibitem{er=epr} J. Maldacena and L. Susskind, \textit{Fortsch. Phys.} 61, 781 (2013).
\bibitem{vanraamsdonk} M. Van Raamsdonk, \textit{Int. J. Mod. Phys. D} 19, 2429 (2010).
\bibitem{ryu-takayanagi} S. Ryu and T. Takayanagi, \textit{Phys. Rev. Lett.} 96, 181602 (2006).
\bibitem{cao2017} C. Cao et al., \textit{Phys. Rev. D} 95, 024031 (2017).
\end{thebibliography}

\end{document}
