\documentclass[11pt,a4paper]{article}

% -------------------- Encodage / Langue --------------------
\usepackage[utf8]{inputenc}
\usepackage[T1]{fontenc}
\usepackage[french]{babel}

% -------------------- Maths / Symboles --------------------
\usepackage{amsmath,amssymb,amsfonts}
\usepackage{bm}
\usepackage{braket}

% -------------------- Mise en page / Figures --------------------
\usepackage{geometry}
\usepackage{graphicx}
\usepackage{caption}
\usepackage{setspace}
\usepackage{booktabs}
\usepackage{authblk}
\usepackage{tcolorbox}
\usepackage{xcolor}
\usepackage{siunitx}

% -------------------- Hyperliens (en dernier) --------------------
\usepackage{hyperref}

\geometry{margin=2.5cm}
\setstretch{1.15}

\definecolor{myblue}{RGB}{0,82,147}
\definecolor{mygreen}{RGB}{0,128,0}
\definecolor{myred}{RGB}{180,0,0}

\hypersetup{
  colorlinks=true,
  linkcolor=myblue,
  citecolor=myblue,
  urlcolor=myblue
}

% Figures : le .tex est dans paper/ donc ../figures marche
\graphicspath{{../figures/}{./figures/}{figures/}}

\title{\textbf{Bottom-Up Quantum Gravity :}\\
\vspace{0.3em}
\Large{Une Théorie de l'Émergence Gravitationnelle}\\
\vspace{0.3em}
\large{Espace, Temps et Gravité depuis l'Intrication Quantique}}

\author{Farid Hamdad}
\date{Février 2026}

\begin{document}
\maketitle

\vspace{0.8em}
\begin{center}
\itshape
L'intrication est une structure géométrique.\\
Ce que nous appelons espace est le motif collectif de ces corrélations.\\
Ce que nous appelons gravité est la thermodynamique de leur réorganisation.
\end{center}
\vspace{1.2em}

\begin{abstract}
\begin{tcolorbox}[colback=gray!10,colframe=myblue,title=Résumé Opérationnel]
Nous présentons \textbf{Bottom-Up Quantum Gravity}, un cadre minimal où espace, temps et gravité
émergent de l'intrication d'un état quantique global pur $\ket{\Psi}$ \emph{sans géométrie préalable}.
\textbf{Résultat central :} la dimension spatiale n'est pas postulée mais \emph{mesurée} via reconstruction métrique :
des états à intrication principalement locale se reconstruisent en 2D, tandis que des états à intrication non-locale
imposent une dimension effective plus élevée.
Nous validons numériquement trois signatures sur des systèmes finis :
(1) temps relationnel via flot modulaire, (2) géométrie informationnelle via information mutuelle et embarquement (MDS),
et (3) gravité thermodynamique effective via un test variationnel de type Jacobson.
Nous ajoutons un test opérationnel de type ER=EPR : des paires topologiquement distantes deviennent géométriquement proches
dans l'espace émergent lorsque l'intrication non-locale est renforcée.
Les résultats sont établis sur $N=9$ qubits puis confirmés sur $N=16$, réduisant les effets de taille finie et renforçant la robustesse.
\end{tcolorbox}
\end{abstract}

%=============================================================================
\section{Ontologie Minimale : Le Postulat Fondamental}
%=============================================================================

\begin{tcolorbox}[colback=myblue!5,colframe=myblue,title=Postulat Unique]
Il existe un état quantique global pur $\ket{\Psi} \in \mathcal{H} = \bigotimes_{i \in V} \mathcal{H}_i$
où $V$ indexe un ensemble fini de degrés de liberté élémentaires (qubits).
\end{tcolorbox}

\textbf{Ce qui est absent :}
\begin{itemize}
  \item aucun espace-temps préexistant ;
  \item aucune métrique a priori ;
  \item aucune structure causale externe ;
  \item aucun champ classique fondamental.
\end{itemize}

L'état $\ket{\Psi}$ est \textbf{atemporel} au sens où il n'évolue pas par rapport à un temps externe.
Toute dynamique doit émerger de relations internes, i.e. d'observables de sous-systèmes et de corrélations entre eux.
L'approche est volontairement \emph{bottom-up} : partir de $\ket{\Psi}$ et extraire des notions opératoires de temps,
distance, dimension et gravité effective.

%=============================================================================
\section{Systèmes Modèles et Protocole Numérique Unifié}
\label{sec:systeme}
%=============================================================================

Nous adoptons un protocole unique réutilisé à différentes tailles $N$ :
(i) définir une famille d'états $\ket{\Psi(\lambda)}$ contrôlée par un paramètre de non-localité corrélationnelle $\lambda$,
(ii) calculer l'information mutuelle $I(i:j)$, (iii) définir une distance informationnelle $d_{ij}$,
(iv) reconstruire une géométrie par embarquement (MDS), et (v) tester des signatures thermodynamiques (Jacobson)
et géodésiques (ER=EPR).

\subsection{Tailles étudiées : $N=9$ puis $N=16$}

\begin{itemize}
  \item \textbf{$N=9$ :} grille $3\times 3$ (preuve de concept exacte, $2^9=512$).
  \item \textbf{$N=16$ :} grille $4\times 4$ (validation intermédiaire, réduction des effets de bord).
\end{itemize}

Les résultats $N=9$ fixent la structure conceptuelle ; l'extension à $N=16$ teste la robustesse.

\subsection{Famille d'états paramétrée par $\lambda$}

Nous considérons une famille d'états contrôlée par $\lambda \in [0,1]$ :
\begin{equation}
\ket{\Psi(\lambda)} = \frac{1}{\mathcal{N}(\lambda)} \sum_{c=0}^{2^N-1} \alpha_c(\lambda)\, \ket{c}.
\end{equation}

Intuition :
\begin{itemize}
  \item $\lambda=0$ : intrication dominée par la localité sur la grille ;
  \item $\lambda \to 1$ : intrication renforcée entre sites topologiquement distants (coins).
\end{itemize}

Une construction explicite est fournie en Annexe~\ref{app:construction-etats}.

\subsection{Hamiltonien de référence (pour tests thermodynamiques)}

Pour les tests de type Jacobson, nous utilisons un Ising transverse perturbé et un terme non-local entre coins :
\begin{equation}
H(\lambda) = -J \sum_{\langle i,j \rangle} Z_i Z_j
             - h(\lambda) \sum_i X_i
             - \lambda\, J_{\text{nl}} \sum_{\text{coins}} X_i X_j.
\label{eq:hamiltonien}
\end{equation}

Dans la pratique, $\lambda$ doit être compris comme un bouton contrôlant la \emph{non-localité corrélationnelle}.

%=============================================================================
\section{Temps Émergent : Flot Modulaire}
%=============================================================================

\subsection{Partition et horloge relationnelle}

On partitionne :
\begin{equation}
\mathcal{H} = \mathcal{H}_C \otimes \mathcal{H}_S,
\end{equation}
où $C$ est un sous-système horloge, $S$ le reste.
En l'absence de temps externe, on adopte une lecture relationnelle (Page--Wootters) :
\begin{equation}
(\hat{H}_C + \hat{H}_S)\ket{\Psi} = 0,
\end{equation}
où la dynamique est encodée dans les corrélations horloge-système.

\subsection{Flot modulaire}

Pour toute région $A$ :
\begin{equation}
\rho_A(\lambda) = \mathrm{Tr}_{\bar{A}} \ket{\Psi(\lambda)}\bra{\Psi(\lambda)},
\qquad
K_A(\lambda) = -\log \rho_A(\lambda).
\end{equation}

Le flot modulaire définit une évolution interne :
\begin{equation}
\mathcal{O}(\tau) = e^{i K_A(\lambda)\tau}\, \mathcal{O}\, e^{-i K_A(\lambda)\tau}.
\end{equation}

Interprétation : le temps est un paramètre opératoire déduit de l'état via l'intrication du sous-système.

%=============================================================================
\section{Espace Émergent : Géométrie Informationnelle}
%=============================================================================

\subsection{Information mutuelle et distance}

L'information mutuelle entre sites définit une proximité émergente :
\begin{equation}
I_\lambda(i:j) = S(\rho_i) + S(\rho_j) - S(\rho_{ij}),
\qquad
S(\rho) = -\mathrm{Tr}(\rho \log \rho).
\end{equation}

Nous définissons une distance informationnelle :
\begin{equation}
d_{ij}(\lambda) =
-\log \left(\frac{I_\lambda(i:j)}{I_\lambda^{\max} + \epsilon}\right),
\label{eq:distance}
\end{equation}
avec $\epsilon = 10^{-12}$ pour régulariser.

\subsection{Embarquement (MDS) et erreur}

Pour chaque $\lambda$, on reconstruit un ensemble de points $x_i(\lambda) \in \mathbb{R}^d$ par MDS.
L'erreur relative :
\begin{equation}
\epsilon_d(\lambda) =
\sqrt{\frac{\sum_{i<j} \left(\|x_i - x_j\| - d_{ij}(\lambda)\right)^2}
{\sum_{i<j} d_{ij}(\lambda)^2}}.
\end{equation}

La \emph{dimension effective} est la plus petite dimension $d$ pour laquelle $\epsilon_d(\lambda)$ devient
faible et stable (à incertitudes numériques près).

%=============================================================================
\section{Gravité Émergente : Test Thermodynamique de Jacobson}
\label{sec:jacobson}
%=============================================================================

\subsection{Protocole}

Pour chaque $\lambda$ :
\begin{itemize}
  \item entropie d'intrication $S_A(\lambda)$ d'une région $A$ ;
  \item énergie $E(\lambda)=\bra{\Psi(\lambda)}H(\lambda)\ket{\Psi(\lambda)}$.
\end{itemize}

Une température inverse effective est estimée par différence finie :
\begin{equation}
\beta(\lambda) \approx
\frac{S_A(\lambda + \delta\lambda) - S_A(\lambda - \delta\lambda)}
{E(\lambda + \delta\lambda) - E(\lambda - \delta\lambda)},
\qquad \delta\lambda>0.
\end{equation}

On teste alors $\delta S \simeq \beta\,\delta E$ comme relation variationnelle effective.

%=============================================================================
\section{Résultats (I) : $N=9$ qubits — Preuve de Concept}
\label{sec:N9}
%=============================================================================

Dans cette section, nous présentons les résultats sur la grille $3\times 3$ ($N=9$),
qui sert de référence minimale entièrement calculable.

\subsection{Transition dimensionnelle à $N=9$}

\begin{table}[h]
\centering
\caption{Erreur de reconstruction MDS $\epsilon_d(\lambda) \times 10^2$ (moyenne $\pm$ écart-type sur 10 initialisations).}
\label{tab:dimensionN9}
\begin{tabular}{@{}lcccc@{}}
\toprule
$\lambda$ & $\epsilon_{2D}$ & $\epsilon_{3D}$ & $\epsilon_{4D}$ & Ratio $\epsilon_{2D}/\epsilon_{3D}$ \\
\midrule
0.0 (local) & $5.9 \pm 0.3$ & $5.8 \pm 0.3$ & $5.7 \pm 0.4$ & $1.02 \pm 0.08$ \\
0.2 & $4.2 \pm 0.4$ & $3.1 \pm 0.3$ & $3.0 \pm 0.4$ & $1.35 \pm 0.15$ \\
0.4 & $2.8 \pm 0.5$ & $1.2 \pm 0.2$ & $1.1 \pm 0.3$ & $2.33 \pm 0.45$ \\
0.6 & $1.5 \pm 0.4$ & $0.4 \pm 0.1$ & $0.4 \pm 0.1$ & $3.75 \pm 1.10$ \\
0.8 & $0.8 \pm 0.2$ & $0.15 \pm 0.05$ & $0.14 \pm 0.06$ & $5.33 \pm 1.80$ \\
1.0 (non-local) & $0.08 \pm 0.03$ & $0.01 \pm 0.005$ & $0.009 \pm 0.004$ & $8.00 \pm 3.50$ \\
\bottomrule
\end{tabular}
\end{table}

\begin{figure}[h!]
\centering
\includegraphics[width=0.95\linewidth]{fig1_dimension_emergente.png}
\caption{
\textbf{Émergence dimensionnelle induite par l’intrication (N=9).}
Reconstructions MDS et erreurs relatives : un état local ($\lambda\simeq 0$) est compatible 2D,
tandis qu’un état non-local ($\lambda\simeq 1$) exige un embarquement 3D pour réduire l’erreur.
}
\label{fig:dimensionN9}
\end{figure}

\subsection{Signature ER=EPR (N=9)}

\begin{table}[h]
\centering
\caption{Test ER=EPR (N=9) pour la paire de coins (0,8).}
\label{tab:ereprN9}
\begin{tabular}{@{}lcccc@{}}
\toprule
$\lambda$ & $I(0:8)$ & $d_{\text{topo}}$ & $d_{\text{bulk}}$ & Compression $C$ \\
\midrule
0.0 & $0.009 \pm 0.001$ & 4 & $4.0 \pm 0.2$ & $1.0 \pm 0.1$ \\
0.5 & $0.45 \pm 0.05$ & 4 & $1.8 \pm 0.3$ & $2.2 \pm 0.4$ \\
1.0 & $1.28 \pm 0.02$ & 4 & $0.04 \pm 0.02$ & $100 \pm 50$ \\
\bottomrule
\end{tabular}
\end{table}

\begin{figure}[h!]
\centering
\includegraphics[width=0.95\linewidth]{fig2_er_epr.png}
\caption{
\textbf{Signature géométrique de type ER=EPR (N=9).}
La paire de sites topologiquement distante devient géométriquement proche dans l'espace reconstruit lorsque l'intrication non-locale augmente.
Signature \emph{wormhole-like} au sens géométrique (système fini, pas de trou de ver dynamique).
}
\label{fig:ereprN9}
\end{figure}

\subsection{Test de Jacobson (N=9)}

\begin{figure}[h!]
\centering
\includegraphics[width=0.95\linewidth]{fig3_jacobson.png}
\caption{
\textbf{Relation thermodynamique de type Jacobson (N=9).}
Relation $S(E)$ et test variationnel $\delta S \simeq \beta\,\delta E$.
Résultat interprété comme une thermodynamique effective de l'intrication dans un système fini.
}
\label{fig:jacobsonN9}
\end{figure}

%=============================================================================
\section{Résultats (II) : $N=16$ qubits — Validation Intermédiaire}
\label{sec:N16}
%=============================================================================

L'objectif ici est de tester que les signatures observées à $N=9$ ne sont pas des artefacts
de petite taille. Nous répétons le protocole sur une grille $4\times 4$ ($N=16$),
ce qui réduit les effets de bord et augmente la distance topologique entre coins opposés.

\subsection{Graphe $4\times 4$ et paires de coins}

\begin{figure}[h!]
\centering
\includegraphics[width=0.70\linewidth]{fig4_grid_N16.png}
\caption{
\textbf{Système $N=16$ (grille $4\times 4$) et coins utilisés pour le test ER=EPR.}
La distance topologique entre coins opposés est $d_{\text{topo}}=6$.
}
\label{fig:grid16}
\end{figure}

\subsection{Transition dimensionnelle (MDS) à $N=16$}

\begin{figure}[h!]
\centering
\includegraphics[width=0.95\linewidth]{fig5_mds_N16.png}
\caption{
\textbf{Reconstruction géométrique MDS pour $N=16$.}
Pour plusieurs valeurs de $\lambda$, la hiérarchie des erreurs montre qu'à faible non-localité
les reconstructions 2D/3D sont comparables, tandis qu'à forte non-localité la reconstruction 3D devient nettement meilleure que 2D.
Cela confirme la transition dimensionnelle observée à $N=9$ et en améliore la stabilité.
}
\label{fig:mds16}
\end{figure}

\subsection{Signature ER=EPR à $N=16$ : raccourci géodésique}

\begin{figure}[h!]
\centering
\includegraphics[width=0.95\linewidth]{fig6_erepr_N16.png}
\caption{
\textbf{Test ER=EPR pour $N=16$ (coins opposés 0--15).}
(\textit{Gauche}) la distance géodésique émergente $d_{\text{bulk}}(0,15)$ décroît avec $\lambda$ alors que $d_{\text{topo}}=6$ reste fixe.
(\textit{Centre}) la compression $C=d_{\text{topo}}/d_{\text{bulk}}$ croît monotone et atteint des valeurs $\mathcal{O}(10)$ à forte non-localité.
(\textit{Droite}) corrélation directe entre information mutuelle et proximité géométrique : la géométrie émergente est pilotée par l'intrication.
}
\label{fig:erepr16}
\end{figure}

\subsection{Relation de Jacobson à $N=16$}

\begin{figure}[h!]
\centering
\includegraphics[width=0.95\linewidth]{fig7_jacobson_N16.png}
\caption{
\textbf{Relation thermodynamique de type Jacobson pour $N=16$.}
(\textit{Gauche}) la relation $S(E)$ est bien approximée par un ajustement linéaire sur l'intervalle exploré.
(\textit{Droite}) la pente effective $\beta=dS/dE$ varie lentement avec $\lambda$ et reste stable,
suggérant une thermodynamique effective robuste de l'intrication lorsque la taille du système augmente.
}
\label{fig:jacobson16}
\end{figure}

%=============================================================================
\section{Effets de Taille Finie et Extrapolation}
\label{sec:taille-finie}
%=============================================================================

Le passage de $N=9$ à $N=16$ réduit déjà deux limites majeures :
(i) domination des effets de bord et (ii) faible résolution géométrique.
Néanmoins, les résultats restent ceux de systèmes finis, et doivent être interprétés comme des signatures opératoires.

\begin{table}[h]
\centering
\caption{Sources d'erreur typiques et impacts qualitatifs.}
\label{tab:erreurs}
\begin{tabular}{@{}lcc@{}}
\toprule
Source & Ordre de grandeur & Impact \\
\midrule
Effets de bord & important à $N=9$, réduit à $N=16$ & biais sur distances reconstruites \\
Sensibilité MDS (init.) & $\sim 10\%$ à $15\%$ & dispersion de $\epsilon_d$ \\
Choix de régularisation $\epsilon$ & faible & stabilité des zéros de $I(i:j)$ \\
Systèmes finis & structurel & pas de limite continue ni RG complet \\
\bottomrule
\end{tabular}
\end{table}

Ces tendances motivent l'extension vers $N\gtrsim 25$ (méthodes approximatives, tensor networks, etc.)
pour évaluer la convergence quantitative dans des régimes moins dominés par la finitude.

%=============================================================================
\section{Synthèse et Conclusion}
%=============================================================================

\begin{tcolorbox}[colback=mygreen!5,colframe=mygreen,title=Chaîne d'Émergence Validée]
\[
\ket{\Psi(\lambda)} \xrightarrow{\mathrm{Tr}_{\bar{A}}} \rho_A(\lambda)
\xrightarrow{-\log} K_A(\lambda) \xrightarrow{\text{flot}} \textbf{Temps}
\]
\[
\ket{\Psi(\lambda)} \xrightarrow{I_\lambda(i:j)} d_{ij}(\lambda)
\xrightarrow{\text{MDS}} \textbf{Géométrie} \xrightarrow{\text{tests}} \textbf{Gravité effective}
\]
\[
\lambda \uparrow : \quad \text{dimension effective augmente, et la signature ER=EPR s'active}
\]
\end{tcolorbox}

\textbf{Résultats principaux :}
\begin{enumerate}
  \item \textbf{Dimension émergente :} la reconstruction MDS montre une transition contrôlée par $\lambda$
  où des corrélations non-locales imposent une dimension effective plus élevée.
  \item \textbf{ER=EPR (signature géométrique) :} des paires topologiquement distantes deviennent proches
  dans l'espace émergent lorsque l'intrication non-locale augmente (raccourci géodésique effectif).
  \item \textbf{Jacobson (thermodynamique) :} la relation variationnelle $\delta S \simeq \beta\,\delta E$
  apparaît comme une loi d'état effective de l'intrication.
\end{enumerate}

\textbf{Validation par montée en taille :}
le passage de $N=9$ à $N=16$ confirme que les signatures précédentes persistent et gagnent en stabilité,
ce qui réduit la probabilité d'artefacts purement liés à la petite taille.
L'étape suivante consiste à explorer des tailles $N\gtrsim 25$ avec des méthodes approchées,
afin d'étudier une convergence quantitative vers un régime collectif.

%=============================================================================
\section*{Données et Reproductibilité}
%=============================================================================
Simulations réalisées avec Python/NumPy/SciPy/scikit-learn.
Code et figures : \url{https://github.com/Farid-Hamdad/Bottom-Up-Quantum-Gravity}.

\appendix

\section{Construction des États $\ket{\Psi(\lambda)}$}
\label{app:construction-etats}

Une construction simple utilisée pour générer $\alpha_c(\lambda)$ consiste à biaiser une distribution
d'amplitudes vers des configurations corrélées localement et, lorsque $\lambda$ augmente,
à renforcer des corrélations entre coins :
\begin{equation}
\alpha_c(\lambda) =
\exp\left[ -\beta E_{\text{local}}(c) - \lambda \beta E_{\text{nl}}(c) \right],
\qquad
\mathcal{N}(\lambda)=\sqrt{\sum_c |\alpha_c(\lambda)|^2}.
\end{equation}

où, par exemple :
\begin{align}
E_{\text{local}}(c) &= -\sum_{\langle i,j \rangle} (-1)^{c_i + c_j}, \\
E_{\text{nl}}(c) &= -\sum_{(i,j)\in \text{coins}} (-1)^{c_i + c_j}.
\end{align}

Cette paramétrisation n'est pas unique : elle sert de \emph{générateur contrôlé}
pour explorer la transition local/non-local.

\begin{thebibliography}{9}
\bibitem{jacobson} T. Jacobson, \textit{Phys. Rev. Lett.} 75, 1260 (1995).
\bibitem{er=epr} J. Maldacena and L. Susskind, \textit{Fortsch. Phys.} 61, 781 (2013).
\bibitem{vanraamsdonk} M. Van Raamsdonk, \textit{Int. J. Mod. Phys. D} 19, 2429 (2010).
\bibitem{ryu-takayanagi} S. Ryu and T. Takayanagi, \textit{Phys. Rev. Lett.} 96, 181602 (2006).
\bibitem{cao2017} C. Cao et al., \textit{Phys. Rev. D} 95, 024031 (2017).
\end{thebibliography}

\end{document}

