\documentclass[11pt,a4paper]{article}

% -------------------- Encodage / Langue --------------------
\usepackage[utf8]{inputenc}
\usepackage[T1]{fontenc}
\usepackage[french]{babel}

% -------------------- Maths / Symboles --------------------
\usepackage{amsmath,amssymb,amsfonts}
\usepackage{bm}
\usepackage{braket}

% -------------------- Mise en page / Figures --------------------
\usepackage{geometry}
\usepackage{graphicx}
\usepackage{caption}
\usepackage{setspace}
\usepackage{booktabs}
\usepackage{authblk}
\usepackage{tcolorbox}
\usepackage{xcolor}
\usepackage{siunitx}

% -------------------- Hyperliens (en dernier) --------------------
\usepackage{hyperref}

\geometry{margin=2.5cm}
\setstretch{1.15}

\definecolor{myblue}{RGB}{0,82,147}
\definecolor{mygreen}{RGB}{0,128,0}
\definecolor{myred}{RGB}{180,0,0}

\hypersetup{
  colorlinks=true,
  linkcolor=myblue,
  citecolor=myblue,
  urlcolor=myblue
}

% Figures : le .tex est dans paper/ donc ../figures marche
\graphicspath{{../figures/}{./figures/}{figures/}}

\title{\textbf{Bottom-Up Quantum Gravity :}\\
\vspace{0.3em}
\Large{Une Théorie de l'Émergence Gravitationnelle}\\
\vspace{0.3em}
\large{Espace, Temps et Gravité depuis l'Intrication Quantique}}

\author{Farid Hamdad}
\date{Février 2026}

\begin{document}
\maketitle

\vspace{0.8em}
\begin{center}
\itshape
L'intrication est une structure géométrique.\\
Ce que nous appelons espace est le motif collectif de ces corrélations.\\
Ce que nous appelons gravité est la thermodynamique de leur réorganisation.
\end{center}
\vspace{1.2em}

\begin{abstract}
\begin{tcolorbox}[colback=gray!10,colframe=myblue,title=Résumé Opérationnel]
Nous présentons \textbf{Bottom-Up Quantum Gravity}, un cadre minimal où espace, temps et gravité
émergent de l'intrication d'un état quantique global pur $\ket{\Psi}$ \emph{sans géométrie préalable}.
\textbf{Résultat central :} la dimension spatiale n'est pas postulée mais \emph{mesurée} via reconstruction métrique :
des états à intrication principalement locale se reconstruisent en 2D, tandis que des états à intrication non-locale
imposent une dimension effective plus élevée.
Nous validons numériquement plusieurs signatures sur des systèmes finis :
(1) temps relationnel via flot modulaire, (2) géométrie informationnelle via information mutuelle et embarquement (MDS),
(3) gravité thermodynamique effective via un test variationnel de type Jacobson,
(4) géométrie \emph{intrinsèque} via la dimension spectrale $d_s(\tau)$ du Laplacien du graphe d'intrication,
et (5) une loi de réponse \emph{Einstein-like} au sens opérationnel :
relation locale entre variations d'entropie de coupure (proxy d'aire), courbure discrète (Ollivier--Ricci)
et intensité de perturbation, avec Wasserstein-1 calculé par programmation linéaire.
Nous ajoutons un test opérationnel de type ER=EPR : des paires topologiquement distantes deviennent géométriquement proches
dans l'espace émergent lorsque l'intrication non-locale est renforcée.
\textbf{Nouveau résultat :} l'analyse spectrale du Hamiltonien modulaire $K_A=-\log\rho_A$ révèle un régime chaotique de type RMT
et un plateau du Spectral Form Factor vérifiant $g_2^{\mathrm{plateau}}\sim 1/d_A$, avec un préfacteur
$C=d_A g_2^{\mathrm{plateau}}$ modulé par la topologie (chaîne 1D, grille 2D, graphe aléatoire).
Les résultats sont établis sur $N=9$ qubits puis confirmés sur $N=16$, et complétés par l'étude modulaire à $d_A=256$.
Nous ajoutons enfin une \textbf{détection d'horizon bottom-up} (N=16) : une région BH-like de taille fixée $|A|=N/2$ présente un \emph{bottleneck informationnel} (conductance faible) et une cohésion interne extrême, validé par benchmark vs régions aléatoires, et reproduit par une hiérarchie HIE initialisée par \textbf{communautés Louvain} (rang $2/55$ en mode horizon-aware).
\end{tcolorbox}
\end{abstract}

%=============================================================================
\section{Ontologie Minimale : Le Postulat Fondamental}
%=============================================================================

\begin{tcolorbox}[colback=myblue!5,colframe=myblue,title=Postulat Unique]
Il existe un état quantique global pur $\ket{\Psi} \in \mathcal{H} = \bigotimes_{i \in V} \mathcal{H}_i$
où $V$ indexe un ensemble fini de degrés de liberté élémentaires (qubits).
\end{tcolorbox}

\textbf{Ce qui est absent :}
\begin{itemize}
  \item aucun espace-temps préexistant ;
  \item aucune métrique a priori ;
  \item aucune structure causale externe ;
  \item aucun champ classique fondamental.
\end{itemize}

L'état $\ket{\Psi}$ est \textbf{atemporel} au sens où il n'évolue pas par rapport à un temps externe.
Toute dynamique doit émerger de relations internes, i.e. d'observables de sous-systèmes et de corrélations entre eux.
L'approche est volontairement \emph{bottom-up} : partir de $\ket{\Psi}$ et extraire des notions opératoires de temps,
distance, dimension et gravité effective.

%=============================================================================
\section{Systèmes Modèles et Protocole Numérique Unifié}
\label{sec:systeme}
%=============================================================================

Nous adoptons un protocole unique réutilisé à différentes tailles $N$ :
(i) définir une famille d'états $\ket{\Psi(\lambda)}$ contrôlée par un paramètre de non-localité corrélationnelle $\lambda$,
(ii) calculer l'information mutuelle $I(i:j)$, (iii) définir une distance informationnelle $d_{ij}$,
(iv) reconstruire une géométrie par embarquement (MDS), (v) tester des signatures thermodynamiques (Jacobson)
et géodésiques (ER=EPR), (vi) extraire une dimension intrinsèque via la géométrie spectrale,
(vii) tester une loi de réponse locale liant entropie de coupure et courbure discrète,
et (viii) analyser le flot modulaire via ses signatures spectrales (chaos RMT, SFF).

\subsection{Tailles étudiées : $N=9$ puis $N=16$}

\begin{itemize}
  \item \textbf{$N=9$ :} grille $3\times 3$ (preuve de concept exacte, $2^9=512$).
  \item \textbf{$N=16$ :} grille $4\times 4$ (validation intermédiaire, réduction des effets de bord).
\end{itemize}

\subsection{Famille d'états paramétrée par $\lambda$}

Nous considérons une famille d'états contrôlée par $\lambda \in [0,1]$ :
\begin{equation}
\ket{\Psi(\lambda)} = \frac{1}{\mathcal{N}(\lambda)} \sum_{c=0}^{2^N-1} \alpha_c(\lambda)\, \ket{c}.
\end{equation}

Intuition :
\begin{itemize}
  \item $\lambda=0$ : intrication dominée par la localité sur la grille ;
  \item $\lambda \to 1$ : intrication renforcée entre sites topologiquement distants.
\end{itemize}

Une construction explicite est fournie en Annexe~\ref{app:construction-etats}.

\subsection{Hamiltonien de référence (tests thermodynamiques)}

Pour les tests de type Jacobson, nous utilisons un Ising transverse perturbé et un terme non-local :
\begin{equation}
H(\lambda) = -J \sum_{\langle i,j \rangle} Z_i Z_j
             - h(\lambda) \sum_i X_i
             - \lambda\, J_{\text{nl}} \sum_{\text{coins}} X_i X_j.
\label{eq:hamiltonien}
\end{equation}

%=============================================================================
\section{Temps Émergent : Horloge Relationnelle et Flot Modulaire}
\label{sec:time_modular}
%=============================================================================

\subsection{Partition et horloge relationnelle}

On partitionne :
\begin{equation}
\mathcal{H} = \mathcal{H}_C \otimes \mathcal{H}_S,
\end{equation}
où $C$ est un sous-système horloge, $S$ le reste.
En l'absence de temps externe, on adopte une lecture relationnelle (Page--Wootters) :
\begin{equation}
(\hat{H}_C + \hat{H}_S)\ket{\Psi} = 0,
\end{equation}
où la dynamique est encodée dans les corrélations horloge-système.

\subsection{Flot modulaire}

Pour toute région $A$ :
\begin{equation}
\rho_A(\lambda) = \mathrm{Tr}_{\bar{A}} \ket{\Psi(\lambda)}\bra{\Psi(\lambda)},
\qquad
K_A(\lambda) = -\log \rho_A(\lambda).
\end{equation}

Le flot modulaire définit une évolution interne :
\begin{equation}
\mathcal{O}(\tau) = e^{i K_A(\lambda)\tau}\, \mathcal{O}\, e^{-i K_A(\lambda)\tau}.
\end{equation}

Interprétation : le temps est un paramètre opératoire déduit de l'état via l'intrication du sous-système.

%=============================================================================
\section{Chaos Modulaire : Statistiques Spectrales et Universalité}
\label{sec:modular_chaos}
%=============================================================================

Au-delà de l'interprétation conceptuelle du temps relationnel, nous analysons la structure spectrale de
\(K_A=-\log\rho_A\) afin de tester l'émergence d'un régime chaotique universel de type Random Matrix Theory (RMT).

\subsection{Ratio des gaps}

Soit \(\{\kappa_n\}\) le spectre ordonné de \(K_A\) (après filtrage numérique des niveaux nuls).
On définit les espacements \(\Delta_n=\kappa_{n+1}-\kappa_n\) et le ratio local
\begin{equation}
r_n=\frac{\min(\Delta_n,\Delta_{n+1})}{\max(\Delta_n,\Delta_{n+1})},
\qquad
\langle r\rangle = \frac{1}{N_r}\sum_n r_n.
\end{equation}
Valeurs universelles :
\(
\langle r\rangle_{\text{Poisson}}\simeq 0.386
\),
\(
\langle r\rangle_{\text{GOE}}\simeq 0.536
\),
\(
\langle r\rangle_{\text{GUE}}\simeq 0.603
\).

\subsection{Spectral Form Factor (SFF) et unfolding}

Nous évaluons le \emph{Spectral Form Factor} à partir des niveaux \emph{unfolded} \(\{\tilde{\kappa}_n\}\) :
\begin{equation}
g_2(t)=\frac{1}{d_A^2}\left|\sum_{n=1}^{d_A} e^{-it\,\tilde{\kappa}_n}\right|^2,
\label{eq:sff_def}
\end{equation}
où \(d_A=2^{|A|}\) est la dimension du sous-système.
Le SFF présente typiquement une structure \emph{dip--ramp--plateau} en régime chaotique.

\subsection{Scaling du plateau et constante modulaire}

Dans le régime fortement intriqué, nous observons numériquement :
\begin{equation}
g_2^{\mathrm{plateau}} \sim \frac{1}{d_A}.
\label{eq:plateau_scaling}
\end{equation}
Nous introduisons la constante modulaire :
\begin{equation}
C \equiv d_A\, g_2^{\mathrm{plateau}}.
\label{eq:C_def}
\end{equation}
Le scaling \(1/d_A\) est robuste, mais le préfacteur \(C\) conserve une dépendance à la structure du graphe
d'intrication (topologie / connectivité effective).

\subsection{Dépendance topologique (dA = 256, \(\lambda=0.8\))}

Nous comparons trois topologies (moyenne sur seeds) :
\begin{table}[h!]
\centering
\caption{Chaos modulaire et dépendance topologique de \(C=d_A g_2^{\mathrm{plateau}}\) (dA=256, \(\lambda=0.8\)).}
\label{tab:topology_modular}
\begin{tabular}{@{}lccc@{}}
\toprule
Topologie & \(\langle r\rangle\) & \(g_2^{\mathrm{plateau}}\) & \(C=d_A g_2^{\mathrm{plateau}}\) \\
\midrule
Chaîne 1D            & 0.594 & 0.004739 & 1.213 \\
Grille \(3\times 6\) & 0.575 & 0.005541 & 1.419 \\
ER (p\(\approx 0.176\)) & 0.528 & 0.006351 & 1.626 \\
\bottomrule
\end{tabular}
\end{table}

\subsection{Figures}

\begin{figure}[h!]
\centering
\includegraphics[width=0.90\linewidth]{topology_C.png}
\caption{\textbf{Constante modulaire \(C=d_A g_2^{\mathrm{plateau}}\) selon la topologie} (dA=256, \(\lambda=0.8\)).}
\label{fig:topology_C}
\end{figure}

\begin{figure}[h!]
\centering
\includegraphics[width=0.90\linewidth]{topology_r.png}
\caption{\textbf{Statistiques de niveaux via \(\langle r\rangle\)} (dA=256, \(\lambda=0.8\)).}
\label{fig:topology_r}
\end{figure}

\begin{figure}[h!]
\centering
\includegraphics[width=0.92\linewidth]{topology_SFF.png}
\caption{\textbf{Spectral Form Factor moyen} (unfolded) pour les trois topologies (dip--ramp--plateau).}
\label{fig:topology_SFF}
\end{figure}

\subsection{Interprétation}

Le flot modulaire devient asymptotiquement ergodique (universalité RMT), mais la géométrie d'intrication ne disparaît pas :
elle survit sous forme d'une modulation topologique du préfacteur \(C\). Cela fournit un mécanisme naturel, dans Bottom-Up,
par lequel la structure spatiale émergente influence une dynamique temporelle relationnelle associée au sous-système.

%=============================================================================
\section{Espace Émergent : Géométrie Informationnelle}
\label{sec:space_info}
%=============================================================================

\subsection{Information mutuelle et distance}

L'information mutuelle entre sites définit une proximité émergente :
\begin{equation}
I_\lambda(i:j) = S(\rho_i) + S(\rho_j) - S(\rho_{ij}),
\qquad
S(\rho) = -\mathrm{Tr}(\rho \log \rho).
\end{equation}

Nous définissons une distance informationnelle :
\begin{equation}
d_{ij}(\lambda) =
-\log \left(\frac{I_\lambda(i:j)}{I_\lambda^{\max} + \epsilon}\right),
\label{eq:distance}
\end{equation}
avec $\epsilon = 10^{-12}$ pour régulariser.

\subsection{Embarquement (MDS) et erreur}

Pour chaque $\lambda$, on reconstruit un ensemble de points $x_i(\lambda) \in \mathbb{R}^d$ par MDS.
L'erreur relative :
\begin{equation}
\epsilon_d(\lambda) =
\sqrt{\frac{\sum_{i<j} \left(\|x_i - x_j\| - d_{ij}(\lambda)\right)^2}
{\sum_{i<j} d_{ij}(\lambda)^2}}.
\end{equation}

La \emph{dimension effective} est la plus petite dimension $d$ pour laquelle $\epsilon_d(\lambda)$ devient
faible et stable (à incertitudes numériques près).

%=============================================================================
\section{Gravité Émergente : Test Thermodynamique de Jacobson}
\label{sec:jacobson}
%=============================================================================

\subsection{Protocole}

Pour chaque $\lambda$ :
\begin{itemize}
  \item entropie d'intrication $S_A(\lambda)$ d'une région $A$ ;
  \item énergie $E(\lambda)=\bra{\Psi(\lambda)}H(\lambda)\ket{\Psi(\lambda)}$.
\end{itemize}

Une température inverse effective est estimée par différence finie :
\begin{equation}
\beta(\lambda) \approx
\frac{S_A(\lambda + \delta\lambda) - S_A(\lambda - \delta\lambda)}
{E(\lambda + \delta\lambda) - E(\lambda - \delta\lambda)},
\qquad \delta\lambda>0.
\end{equation}

On teste alors $\delta S \simeq \beta\,\delta E$ comme relation variationnelle effective.

%=============================================================================
\section{Géométrie Spectrale : Laplacien et Dimension Spectrale}
\label{sec:spectral}
%=============================================================================

Au-delà de l'embarquement (MDS), nous introduisons une mesure intrinsèque de dimension
fondée sur le spectre du Laplacien du graphe d'intrication.

\subsection{Graphe d'intrication}

À partir des poids $w_{ij}=I_\lambda(i:j)$, on définit $W=[w_{ij}]$, $D_{ii}=\sum_j w_{ij}$,
et le Laplacien combinatoire
\begin{equation}
L = D - W.
\end{equation}

\subsection{Trace de chaleur et dimension spectrale}

On définit la trace de chaleur :
\begin{equation}
\mathcal{Z}(\tau)=\mathrm{Tr}\,e^{-\tau L}=\sum_{\alpha=1}^{N} e^{-\tau \lambda_\alpha},
\end{equation}
où $\{\lambda_\alpha\}$ sont les valeurs propres de $L$.
La dimension spectrale est :
\begin{equation}
d_s(\tau) = -2\,\frac{d\log \mathcal{Z}(\tau)}{d\log \tau}.
\end{equation}

\subsection{Plateau IR et dimension effective}

Pour $N=9$ et $N=16$, $d_s(\tau)$ présente typiquement un plateau IR sur une fenêtre $(\tau_1,\tau_2)$.
Nous définissons
\begin{equation}
d_{\mathrm{IR}} \equiv \left\langle d_s(\tau)\right\rangle_{\tau\in[\tau_1,\tau_2]}.
\end{equation}
Numériquement, $d_{\mathrm{IR}}$ se situe typiquement dans l'intervalle $\sim 2.5$--$2.7$ selon $\lambda$,
ce qui suggère une géométrie effective non-entière et renforce la lecture ``dimension émergente''.

%=============================================================================
\section{Courbure Discrète et Loi de Réponse Einstein-like}
\label{sec:einstein_like}
%=============================================================================

Cette section introduit une courbure discrète (Ricci) et teste une loi de réponse locale liant
variations d'une fonctionnelle de frontière (proxy d'aire) et variations de courbure.

\subsection{Entropie de coupure comme fonctionnelle de frontière}

Pour une région $A\subset V$, on définit un proxy d'entropie de frontière :
\begin{equation}
S_{\mathrm{cut}}(A)=\sum_{i\in A}\sum_{j\notin A} w_{ij}.
\end{equation}
Cette quantité est corrélée à l'entropie de von Neumann $S(\rho_A)$ tout en étant calculable
directement à partir des poids du graphe.

\subsection{Courbure d'Ollivier--Ricci (Wasserstein-1 exact)}

On associe à chaque nœud $i$ une mesure locale $\mu_i$ sur $V$ :
\begin{equation}
\mu_i(j)=\frac{w_{ij}}{\sum_k w_{ik}}.
\end{equation}
Pour une arête $(i,j)$, la courbure d'Ollivier--Ricci est définie par
\begin{equation}
\kappa_{ij}=1-\frac{W_1(\mu_i,\mu_j)}{d(i,j)},
\end{equation}
où $W_1$ est la distance de Wasserstein-1 calculée par programmation linéaire
(Annexe~\ref{app:W1LP}).
La distance $d(i,j)$ peut être choisie comme distance géodésique de graphe
(longueurs d'arêtes définies à partir de $w_{ij}$) ou distance de résistance via $L^+$.

\subsection{Courbure régionale}

Pour une région $A$, on définit une courbure moyenne :
\begin{equation}
\mathcal{R}(A)\equiv \frac{1}{|E(A)|}\sum_{(u,v)\in E(A)} \kappa_{uv},
\end{equation}
où $E(A)$ est l'ensemble des arêtes internes à la région.

\subsection{Perturbations locales et source}

On considère des perturbations locales des poids $w_{ij}\to w_{ij}+\delta w_{ij}$.
Pour une région $A=B(i,R)$ (boule de rayon $R$ pour une métrique donnée), on définit une intensité de source :
\begin{equation}
\Delta M(A)=\sum_{u\in A}\sum_v |\delta w_{uv}|.
\end{equation}

\subsection{Équation d'état opérationnelle (réponse linéaire)}

Nous testons une relation de réponse locale de type équation d'état :
\begin{equation}
\Delta S_{\mathrm{cut}}(A)=a(R)\,\Delta\mathcal{R}(A)+b(R)\,\Delta M(A)+\varepsilon,
\label{eq:operational_eos}
\end{equation}
où $\Delta$ désigne la différence avant/après perturbation, et $A=B(i,R)$.

La stabilité de $a(R),b(R)$ est évaluée via bootstrap (Annexe~\ref{app:bootstrap}).

%=============================================================================
\section{Résultats (I) : $N=9$ qubits — Preuve de Concept}
\label{sec:N9}
%=============================================================================

Dans cette section, nous présentons les résultats sur la grille $3\times 3$ ($N=9$),
qui sert de référence minimale entièrement calculable.

\subsection{Transition dimensionnelle à $N=9$}

\begin{table}[h]
\centering
\caption{Erreur de reconstruction MDS $\epsilon_d(\lambda) \times 10^2$ (moyenne $\pm$ écart-type sur 10 initialisations).}
\label{tab:dimensionN9}
\begin{tabular}{@{}lcccc@{}}
\toprule
$\lambda$ & $\epsilon_{2D}$ & $\epsilon_{3D}$ & $\epsilon_{4D}$ & Ratio $\epsilon_{2D}/\epsilon_{3D}$ \\
\midrule
0.0 (local) & $5.9 \pm 0.3$ & $5.8 \pm 0.3$ & $5.7 \pm 0.4$ & $1.02 \pm 0.08$ \\
0.2 & $4.2 \pm 0.4$ & $3.1 \pm 0.3$ & $3.0 \pm 0.4$ & $1.35 \pm 0.15$ \\
0.4 & $2.8 \pm 0.5$ & $1.2 \pm 0.2$ & $1.1 \pm 0.3$ & $2.33 \pm 0.45$ \\
0.6 & $1.5 \pm 0.4$ & $0.4 \pm 0.1$ & $0.4 \pm 0.1$ & $3.75 \pm 1.10$ \\
0.8 & $0.8 \pm 0.2$ & $0.15 \pm 0.05$ & $0.14 \pm 0.06$ & $5.33 \pm 1.80$ \\
1.0 (non-local) & $0.08 \pm 0.03$ & $0.01 \pm 0.005$ & $0.009 \pm 0.004$ & $8.00 \pm 3.50$ \\
\bottomrule
\end{tabular}
\end{table}

\begin{figure}[h!]
\centering
\includegraphics[width=0.95\linewidth]{fig1_dimension_emergente.png}
\caption{
\textbf{Émergence dimensionnelle induite par l'intrication (N=9).}
Reconstructions MDS et erreurs relatives : un état local ($\lambda\simeq 0$) est compatible 2D,
tandis qu'un état non-local ($\lambda\simeq 1$) exige un embarquement 3D pour réduire l'erreur.
}
\label{fig:dimensionN9}
\end{figure}

\subsection{Signature ER=EPR (N=9)}

\begin{table}[h]
\centering
\caption{Test ER=EPR (N=9) pour la paire de coins (0,8).}
\label{tab:ereprN9}
\begin{tabular}{@{}lcccc@{}}
\toprule
$\lambda$ & $I(0:8)$ & $d_{\text{topo}}$ & $d_{\text{bulk}}$ & Compression \\
\midrule
0.0 & $0.009 \pm 0.001$ & 4 & $4.0 \pm 0.2$ & $1.0 \pm 0.1$ \\
0.5 & $0.45 \pm 0.05$ & 4 & $1.8 \pm 0.3$ & $2.2 \pm 0.4$ \\
1.0 & $1.28 \pm 0.02$ & 4 & $0.04 \pm 0.02$ & $100 \pm 50$ \\
\bottomrule
\end{tabular}
\end{table}

\begin{figure}[h!]
\centering
\includegraphics[width=0.95\linewidth]{fig2_er_epr.png}
\caption{
\textbf{Signature géométrique de type ER=EPR (N=9).}
La paire de sites topologiquement distante devient géométriquement proche dans l'espace reconstruit lorsque l'intrication non-locale augmente.
Signature \emph{wormhole-like} au sens géométrique (système fini, pas de trou de ver dynamique).
}
\label{fig:ereprN9}
\end{figure}

\subsection{Test de Jacobson (N=9)}

\begin{figure}[h!]
\centering
\includegraphics[width=0.95\linewidth]{fig3_jacobson.png}
\caption{
\textbf{Relation thermodynamique de type Jacobson (N=9).}
Relation $S(E)$ et test variationnel $\delta S \simeq \beta\,\delta E$.
Résultat interprété comme une thermodynamique effective de l'intrication dans un système fini.
}
\label{fig:jacobsonN9}
\end{figure}

%=============================================================================
\section{Résultats (II) : $N=16$ qubits — Validation Intermédiaire}
\label{sec:N16}
%=============================================================================

L'objectif ici est de tester que les signatures observées à $N=9$ ne sont pas des artefacts
de petite taille. Nous répétons le protocole sur une grille $4\times 4$ ($N=16$),
ce qui réduit les effets de bord et augmente la distance topologique entre coins opposés.

\subsection{Graphe $4\times 4$ et paires de coins}

\begin{figure}[h!]
\centering
\includegraphics[width=0.70\linewidth]{fig4_grid_N16.png}
\caption{
\textbf{Système $N=16$ (grille $4\times 4$) et coins utilisés pour le test ER=EPR.}
La distance topologique entre coins opposés est $d_{\text{topo}}=6$.
}
\label{fig:grid16}
\end{figure}

\subsection{Transition dimensionnelle (MDS) à $N=16$}

\begin{figure}[h!]
\centering
\includegraphics[width=0.95\linewidth]{fig5_mds_N16.png}
\caption{
\textbf{Reconstruction géométrique MDS pour $N=16$.}
Pour plusieurs valeurs de $\lambda$, la hiérarchie des erreurs montre qu'à faible non-localité
les reconstructions 2D/3D sont comparables, tandis qu'à forte non-localité la reconstruction 3D devient nettement meilleure que 2D.
Cela confirme la transition dimensionnelle observée à $N=9$ et en améliore la stabilité.
}
\label{fig:mds16}
\end{figure}

\subsection{Signature ER=EPR à $N=16$ : raccourci géodésique}

\begin{figure}[h!]
\centering
\includegraphics[width=0.95\linewidth]{fig6_erepr_N16.png}
\caption{
\textbf{Test ER=EPR pour $N=16$ (coins opposés 0--15).}
(\textit{Gauche}) la distance géodésique émergente $d_{\text{bulk}}(0,15)$ décroît avec $\lambda$ alors que $d_{\text{topo}}=6$ reste fixe.
(\textit{Centre}) la compression croît monotone et atteint des valeurs $\mathcal{O}(10)$ à forte non-localité.
(\textit{Droite}) corrélation directe entre information mutuelle et proximité géométrique : la géométrie émergente est pilotée par l'intrication.
}
\label{fig:erepr16}
\end{figure}

\subsection{Relation de Jacobson à $N=16$}

\begin{figure}[h!]
\centering
\includegraphics[width=0.95\linewidth]{fig7_jacobson_N16.png}
\caption{
\textbf{Relation thermodynamique de type Jacobson pour $N=16$.}
(\textit{Gauche}) la relation $S(E)$ est bien approximée par un ajustement linéaire sur l'intervalle exploré.
(\textit{Droite}) la pente effective $\beta=dS/dE$ varie lentement avec $\lambda$ et reste stable,
suggérant une thermodynamique effective robuste de l'intrication lorsque la taille du système augmente.
}
\label{fig:jacobson16}
\end{figure}

%=============================================================================
\subsection{Détection d'horizon bottom-up (N=16) : région BH-like et candidats HIE guidés par Louvain}
\label{sec:horizon_detection}
%=============================================================================

\paragraph{Graphe d'intrication à densité fixée.}
À partir de la matrice \(W_{ij}=I(i:j)\), nous construisons un graphe pondéré \(A\) en imposant une densité cible \(\rho\)
par seuillage au quantile (ici \(\rho=1/3\)). Ce choix fixe le nombre d'arêtes et facilite les comparaisons entre seeds.

\paragraph{Métriques de région.}
Pour une région \(A\subset V\) de taille fixée \(|A|=N/2\) (ici \(|A|=8\)), nous mesurons :
\begin{align}
S(A) &= S(\rho_A), \\
\mathrm{cut}(A) &= \sum_{i\in A}\sum_{j\notin A} A_{ij}, \\
\mathrm{internal}(A) &= \sum_{i<j\in A} A_{ij}, \\
\phi(A) &= \frac{\mathrm{cut}(A)}{\min(\mathrm{vol}(A),\mathrm{vol}(\bar{A}))},
\qquad
\mathrm{vol}(A)=\sum_{i\in A}\sum_j A_{ij}.
\end{align}
Dans cette lecture, un \emph{horizon-like bottleneck} correspond à une \(\phi(A)\) faible et une cohésion interne élevée.

\paragraph{Recherche BH-like (optimisation).}
Nous maximisons un score normalisé par z-score (pool de régions aléatoires de même taille) :
\begin{equation}
\mathrm{Score}_{BH}(A)=
w_S\, z(S(A)) - w_\phi\, z(\phi(A)) + w_{\mathrm{int}}\, z(\mathrm{internal}(A)).
\label{eq:bh_score}
\end{equation}
L'optimisation combine échantillonnage aléatoire, hill-climb et acceptation Metropolis.

\paragraph{Résultat BH-like et benchmark.}
Sur \(N=16\) (seed représentative), nous obtenons une région candidate :
\[
A_{BH}=\{0,2,3,5,6,10,11,15\}.
\]
Les métriques et percentiles (comparées à \(3000\) régions aléatoires de taille 8) sont reportées dans les Tableaux~\ref{tab:bh_metrics_N16}
et~\ref{tab:bh_percentiles_N16}. La signature horizon-like est nette : \(\phi\) est très faible (bottleneck)
tandis que \(\mathrm{internal}\) est extrême.

\begin{table}[h!]
\centering
\caption{Région BH-like détectée (N=16, \(|A|=8\), \(\rho=1/3\)) : métriques brutes.}
\label{tab:bh_metrics_N16}
\begin{tabular}{@{}lccccc@{}}
\toprule
Région \(A_{BH}\) & \(S(A)\) & \(\mathrm{cut}(A)\) & \(\mathrm{internal}(A)\) & \(\phi(A)\) & \(\langle r\rangle(K_A)\) \\
\midrule
\(\{0,2,3,5,6,10,11,15\}\) & 7.1667 & 3.2098 & 5.0348 & 0.3951 & 0.6041 \\
\bottomrule
\end{tabular}
\end{table}

\begin{table}[h!]
\centering
\caption{Percentiles de la région BH-like vs \(3000\) régions aléatoires de taille \(|A|=8\) (N=16). Un bottleneck correspond à un percentile \(\phi\) faible.}
\label{tab:bh_percentiles_N16}
\begin{tabular}{@{}lccccc@{}}
\toprule
Mesure & \(S(A)\) & \(\mathrm{cut}(A)\) & \(\mathrm{internal}(A)\) & \(\phi(A)\) & \(\langle r\rangle(K_A)\) \\
\midrule
Percentile (\%) & 23.7 & 1.4 & 98.8 & 2.7 & 62.0 \\
\bottomrule
\end{tabular}
\end{table}

\paragraph{HIE à taille fixée et graines Louvain.}
Nous construisons ensuite une liste de régions candidates \(|A|=8\) via une hiérarchie HIE,
\emph{seedée} par des communautés détectées par Louvain sur le graphe \(A\) (puis raffinées par swaps Metropolis).
Deux classements sont comparés :
(i) \emph{classic} (pondérations équilibrées), et (ii) \emph{horizon-aware} (poids plus fort sur \(\phi\) et pénalité cut).

\paragraph{Résultat-clé : horizon-aware.}
Dans l'exemple représentatif, le rang de \(A_{BH}\) parmi les \(55\) candidats HIE passe de \(12/55\) (classic) à \(2/55\) (horizon-aware),
montrant que la détection par communautés peut récupérer la région BH-like dès lors que le critère encode explicitement le bottleneck.

\begin{figure}[h!]
\centering
\includegraphics[width=0.95\linewidth]{comparison_bh_hie8.png}
\caption{\textbf{Comparaison BH-like vs candidats HIE (N=16, \(|A|=8\)).}
La région
